%!TEX program = xelatex
\documentclass[UTF8,zihao=5]{ctexart}


\title{{\bfseries 第四次作业}}
\author{周涵宇 2018011600}
\date{}

\usepackage[a4paper]{geometry}
\geometry{left=0.75in,right=0.75in,top=1in,bottom=1in}

\usepackage[
UseMSWordMultipleLineSpacing,
MSWordLineSpacingMultiple=1.5
]{zhlineskip}

\usepackage{fontspec}
\setmainfont{Cambria Math}
% \setmonofont{JetBrains Mono}
\setCJKmainfont{仿宋}[AutoFakeBold=true]
\setCJKsansfont{黑体}[AutoFakeBold=true]

\usepackage{bm}
\usepackage{amsmath,amsfonts}
\usepackage{array}

\newcommand{\trans}[0]{^\mathrm{T}}
\newcommand{\tran}[1]{#1^\mathrm{T}}
\newcommand{\hermi}[0]{^\mathrm{H}}

\begin{document}

\maketitle

\subsection*{1}

首先,任何二阶二维反对称张量在标准正交基下,都可表示为:

$$
\bm{A}=b(\bm{e_2e_1-e_1e_2})
$$

同时有
$$
|\bm{A}|=\sqrt{b^2+b^2}=\sqrt{2}|b|
$$

特征方程即为求解:

$$
\lambda^2+b^2=0
$$

则特征值为$\pm i|b|$

使$a=|b|$
则可得所求证。

\subsection*{2}

不妨扩展为三维得到$\bm{e_3}$,其分量都是0。

由于定义好的旋转是唯一的,仅需证明对任意矢量旋转满足旋转的定义,
即可证明其就是唯一的旋转。

设任意二维矢量$\bm{u}=u_i\bm{e_i}$

则$\bm{R}(\theta)=\bm{I}\cos\theta-
\bm{\epsilon}\sin\theta$
可知:

$$
\bm{R\cdot u}=\bm{e_1}(u_1\cos\theta-u_2\sin\theta)
+\bm{e_2}(u_1\sin\theta+u_2\cos\theta)
$$

因此:

$$
||\bm{R\cdot u}|| = u_1^2(\cos^2\theta+\sin^2\theta)
+u_2^2(\cos^2\theta+\sin^2\theta)
+2u_1u_2(\sin\theta\cos\theta-\cos\theta\sin\theta)
=u_1^2+u_2^2=||\bm{u}||
$$

即满足长度不变;


$$
\bm{(R\cdot u)\cdot u}
=(u_1^2+u_2^2)\cos\theta=||\bm{u}||^2\cos\theta
$$

因此夹角大小和$\theta$一样。

$$
\bm{u\times (R\cdot u)}
=\bm{e_3}(u_1^2+u_2^2)\sin\theta
$$

即当$\theta\in[0,\pi]$,上式大于等于0,满足旋转方向的定义。

以上性质是所定义旋转的充分必要条件。

最后讨论其矩阵形式。由于

$$
[\bm{I}]=\begin{bmatrix}
    1&0\\0&1
\end{bmatrix},\ \ 
[\bm{\epsilon}]=\begin{bmatrix}
    0&1\\-1&0
\end{bmatrix},\ \ 
$$

则可知

$$
[\bm{R}(\theta)]=\bm{I}\cos\theta-
\bm{\epsilon}\sin\theta=
\begin{bmatrix}
    \cos\theta&-\sin\theta\\\sin\theta&\cos\theta
\end{bmatrix}
$$

\subsection*{3}

三维反对称张量特征分解可知:
$\theta\bm{L}$为
$$
\bm{V^{-1}\cdot D\cdot V}
$$

其中$\bm{D}=\bm{e_1e_1}i\theta-\bm{e_2e_2}i\theta$

则根据指数函数性质可知

$$
e^{\theta\bm{L}}
=\bm{V^{-1}}e^{\bm{D}}\bm{V}
=\bm{V^{-1}}\bm{M}\bm{V}
$$

其中

$$
\bm{M}=\bm{e_1e_1}(\cos\theta+i\sin\theta)
+\bm{e_2e_2}(\cos\theta-i\sin\theta)
+\bm{e_3e_3}
=\bm{I}+\frac{\sin\theta}{\theta}\bm{D}
+\frac{1-\cos\theta}{\theta^2}\bm{D}^2
$$

则

$$
e^{\theta\bm{L}}
=\bm{I}+\sin\theta\bm{L}+(1-\cos\theta)\bm{L}^2
$$

选取任意矢量$\bm{u}$

则作用于指数形式后:

\begin{equation*}
    \begin{split}
        \bm{v}=e^{\theta\bm{L}}\cdot\bm{u}
        =\bm{u}+\sin\theta(\bm{r}\times\bm{u})
        +(1-\cos\theta)\bm{L}\cdot(\bm{r}\times\bm{u})
        =\bm{u}+\sin\theta(\bm{r}\times\bm{u})
        +(1-\cos\theta)\bm{r}\times(\bm{r}\times\bm{u})
    \end{split}
\end{equation*}

不妨记$\bm{r}$为$\bm{e_3}$,
$\bm{u}$的方向(单位化)为$\bm{e_1}$,
$\bm{r\times u}$的方向(单位化)为$\bm{e_2}$,
$\sqrt{\bm{u\cdot u}}=u$。
$\bm{e_i}$是三维空间的正交标准基。
因此

\begin{equation*}
    \begin{split}
        \bm{v}
        =u\bm{e_1}+u\sin\theta\bm{e_2}
        -u(1-\cos\theta)\bm{e_1}
        =u(\cos\theta \bm{e_1}+\sin\theta \bm{e_2})
    \end{split}
\end{equation*}

以上表达可以验证:

$
\bm{v}\cdot\bm{v}=u
$
,
$
\bm{v}\cdot\bm{u}=u^2\cos\theta
$
,
以及$\bm{u}\times\bm{v}=u^2\sin\theta\bm{e_3}$

根据2的讨论,以上满足三维绕$\bm{r}$旋转$\theta$
的充分必要条件,因此指数形式就是有限转动的精确表达。

\subsection*{4}

此处课上已经通过小转动到有限转动的几何关系证明,此处
按照指数表达形式证明。

$\bm{L}$一定可以特征分解:

$$
\bm{L}=\bm{V^{-1}\cdot E\cdot V},\ \ 
\bm{E}=\bm{e_1}\bm{e_1}i-\bm{e_2}\bm{e_2}i
$$

则

\begin{equation*}
\begin{split}
\lim_{n\rightarrow\infty}{
\left(
    \bm{I}
    +
    \frac{\theta}{n}
    \bm{L}
\right)^n
}
=&
\lim_{n\rightarrow\infty}{
    \bm{V^{-1}}\cdot 
    \left(
        \bm{I}
        +
        \frac{\theta}{n}
        \bm{E}
    \right)^n
    \cdot \bm{V}
}
\\=&
\bm{V^{-1}}\cdot 
\lim_{n\rightarrow\infty}{
    \left(
        \bm{I}
        +
        \frac{\theta}{n}
        \bm{E}
    \right)^n
}\cdot \bm{V}
\\=&
\bm{V^{-1}}\cdot 
\lim_{n\rightarrow\infty}{
    \left(
        \bm{e_1e_1}(1+\frac{i\theta}{n})
        +\bm{e_2e_2}(1-\frac{i\theta}{n})
        +\bm{e_3e_3}(1)
    \right)^n
}\cdot \bm{V}
\\=&
\bm{V^{-1}}\cdot 
    \left(
        \bm{e_1e_1}
        \lim_{n\rightarrow\infty}
        {(1+\frac{i\theta}{n})^n}
        +\bm{e_2e_2}
        \lim_{n\rightarrow\infty}
        {(1-\frac{i\theta}{n})^n}
        +\bm{e_3e_3}(1)
    \right)\cdot \bm{V}
\\=&
\bm{V^{-1}}\cdot 
\left(
    \bm{e_1e_1}
    e^{i\theta}
    +\bm{e_2e_2}
    e^{-i\theta}
    +\bm{e_3e_3}e^0
\right)\cdot \bm{V}
\\=&
\bm{V^{-1}}\cdot 
\exp\left(
    \bm{e_1e_1}
    i\theta
    -\bm{e_2e_2}
    i\theta
\right)\cdot \bm{V}
\\=&
e^{\theta \bm{L}}
\end{split}
\end{equation*}

根据3的结果,以上就是$R(\theta\bm{r})$。

\subsection*{5}

可以按照级数定义考虑:

$$
e^{\bm{A}}=\sum_{n=0}^\infty{
    \frac{\bm{A}^n}{n!}
}
$$

则由于反对称性:

$$
\tran{(e^{\bm{A}})}=\sum_{n=0}^\infty{
    \frac{\bm{A}^n}{n!}(-1)^n
}
$$

则

\begin{equation*}
\begin{split}
    e^{\bm{A}}\cdot\tran{(e^{\bm{A}})}
    =\sum_{m,n=0}^\infty{
        \frac{\bm{A}^{m+n}}{m!n!}(-1)^n
    }
\end{split}
\end{equation*}

讨论:

当$m+n=2k+1,\ k=0,1,2...$,求和内部共有$2k+2$项对应。其中$m,n$互换两两成对,且必有一个
奇数一个偶数。分母相等
而符号相反,求和是0。

当$m+n=2k+2,\ k=0,1,2...$,求和内部共有
$2k+3$项,其中$k+1$项是负号,$k+2$项是正号。

此处考虑二项式定理,当$m+n=2k+2$:

$$
0=(1-1)^{2k+2}=\sum_{m=0}^{2k+2}
\frac{(2k+2)!}{m!n!}1^m(-1)^n
$$

因此以上是0(事实上和$m+n$是奇数还是偶数没有关系)。

综上,级数仅有零次项,即

\begin{equation*}
    \begin{split}
        e^{\bm{A}}\cdot\tran{(e^{\bm{A}})}
        =\bm{I}
\end{split}
\end{equation*}

因此其是正交张量,有

$$
(e^{\bm{A}})^{-1}=e^{\bm{-A}}=\tran{(e^{\bm{A}})}
$$

\end{document}