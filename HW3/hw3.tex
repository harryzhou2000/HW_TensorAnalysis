%!TEX program = xelatex
\documentclass[UTF8,zihao=5]{ctexart}


\title{{\bfseries 第三次作业-2}}
\author{周涵宇 2018011600}
\date{}

\usepackage[a4paper]{geometry}
\geometry{left=0.75in,right=0.75in,top=1in,bottom=1in}

\usepackage[
UseMSWordMultipleLineSpacing,
MSWordLineSpacingMultiple=1.5
]{zhlineskip}

\usepackage{fontspec}
\setmainfont{Cambria Math}
% \setmonofont{JetBrains Mono}
\setCJKmainfont{仿宋}[AutoFakeBold=true]
\setCJKsansfont{黑体}[AutoFakeBold=true]

\usepackage{bm}
\usepackage{amsmath,amsfonts}
\usepackage{array}

\newcommand{\trans}[0]{^\mathrm{T}}
\newcommand{\hermi}[0]{^\mathrm{H}}

\begin{document}

\maketitle

\subsection*{1}

考虑$\bm{A,B,C,D}\in\mathbb{R}^3$,其向量积不妨
在正交标准基$\bm{g_i}$下按照Ricci符号展开:

\begin{equation*}
    \begin{split}
        \bm{E}&=\bm{(A\times B)\times(C\times D)}\\
        &=\bm{g_i}e_{ijk}e_{jst}e_{kmn}A_sB_tC_mD_n\\
    \end{split}
\end{equation*}

Ricci符号满足:

\begin{equation*}
    e_{ijk}e_{ist}=\delta_{js}\delta_{kt}-\delta_{jt}\delta_{ks}
\end{equation*}

则有:

\begin{equation*}
    \begin{split}
        \bm{E}&=\bm{(A\times B)\times(C\times D)}\\
        &=\bm{g_i}(\delta_{ks}\delta_{it}-\delta_{is}\delta_{kt})e_{kmn}A_sB_tC_mD_n
        =\bm{g_i}e_{kmn}A_kB_iC_mD_n-\bm{g_i}e_{tmn}A_iB_tC_mD_n
        =\bm{[A,C,D]B}-\bm{[B,C,D]A}\\
        &=\bm{g_i}(\delta_{im}\delta_{jn}-\delta_{in}\delta_{jm})e_{jst}A_sB_tC_mD_n
        =\bm{g_i}e_{jst}A_sB_tC_iD_j-\bm{g_i}e_{jst}A_sB_tC_jD_i
        =\bm{[A,B,D]C}-\bm{[A,B,C]D}\\
    \end{split}
\end{equation*}

即为所求

\subsection*{2}

已知三维下单位正交基(以下不区分上下标):

$$
\det(\bm{A})=\det([A])=e_{ijk}e_{rst}A_{ir}A_{js}A_{kt}
=e_{ijk}A_{1i}A_{2j}A_{3t}=e_{ijk}A_{i1}A_{j2}A_{t3}
$$

\begin{equation*}
    e^{ijk}e_{rst}=\left|
    \begin{matrix}
        \delta^i_r&\delta^i_s&\delta^i_i\\
        \delta^j_r&\delta^j_s&\delta^j_i\\
        \delta^k_r&\delta^k_s&\delta^k_i\\
    \end{matrix}
    \right|
\end{equation*}

\begin{equation*}
    e^{ijk}e_{rst}e^{rmn}e_{}=\left|
    \begin{matrix}
        \delta^i_r&\delta^i_s&\delta^i_i\\
        \delta^j_r&\delta^j_s&\delta^j_i\\
        \delta^k_r&\delta^k_s&\delta^k_i\\
    \end{matrix}
    \right|
\end{equation*}

则


\begin{equation*}
    \begin{split}
        det(\bm{A})&=
        e_{ijk}e_{rst}A_{ir}A_{js}A_{kt}\\
        &=e_{ijk}e_{rst}
        (B_{i1}C_{1r}+B_{i2}C_{2r}+B_{i3}C_{3r})
        (B_{j1}C_{1s}+B_{j2}C_{2s}+B_{j3}C_{3s})
        (B_{k1}C_{1t}+B_{k2}C_{2t}+B_{k3}C_{3t})\\
    \end{split}
\end{equation*}

观察上式,分别交换右端项中的$ij,jk,ki,rs,st,tr$,同时对
$ijk,rst$分别轮换,得到一系相等的式子,将其求和只剩下6项,发现
其都是乘积中$1,2,3$分别出现两次,整理后得到

\begin{equation*}
    \begin{split}
        det(\bm{A})&=
        e_{ijk}e_{rst}B_{i1}B_{j2}B_{t3}C_{1r}C_{2s}C_{3t}\\
        &=\det(\bm{B})\det(\bm{C})
    \end{split}
\end{equation*}

因此得到所求关系。

\subsection*{3}

不妨找到正交基$\bm{g_i}$是$\bm{B}$的特征向量,对应特征值$\lambda_i$。

\subsubsection*{a}

\begin{equation*}
    \begin{split}
        \bm{[B\cdot u,v,w]}
        =&\left[\sum_i{\lambda_iu_i\bm{g_i}},\bm{v},\bm{w}\right]\\
        =&\lambda_1[u_1\bm{g_1,v,w}]
            +\lambda_2[u_2\bm{g_2,v,w}]
            +\lambda_3[u_3\bm{g_3,v,w}]\\
    \end{split}
\end{equation*}
(上式及以下取消求和约定)

上式$v,w$展开后共27项,与$\bm{[u,B\cdot v,w]},\bm{[u,v,B\cdot w]}$
三组展开求和后,
考察每项$[u_i\bm{g_i},v_j\bm{g_j},v_k\bm{g_k}]$,发现其在第一组展开中
系数是$\lambda_i$,第二、第三组中分别是$\lambda_j,\lambda_k$,易知对于
正交基$i,j,k$不同时才非0,这部分系数都是$I_1$,不妨将其余部分补全为$I_1$的系数,
求和后得到$I_1[\bm{u},\bm{v},\bm{w}]$,即为所求证。

\subsubsection*{b}

\begin{equation*}
    \begin{split}
        \bm{[B\cdot u,B\cdot v,w]}
        =&\left[\sum_i{\lambda_iu_i\bm{g_i}},\sum_i{\lambda_iv_i\bm{g_i}},\bm{w}\right]\\
        =
        &\lambda_1\lambda_1[u_1\bm{g_1},v_1\bm{g_1,w}]
        +\lambda_1\lambda_2[u_1\bm{g_1},v_2\bm{g_2,w}]
        +\lambda_1\lambda_3[u_1\bm{g_1},v_3\bm{g_3,w}]\\+
        &\lambda_2\lambda_1[u_2\bm{g_2},v_1\bm{g_1,w}]
        +\lambda_2\lambda_2[u_2\bm{g_2},v_2\bm{g_2,w}]
        +\lambda_2\lambda_3[u_2\bm{g_2},v_3\bm{g_3,w}]\\+
        &\lambda_3\lambda_1[u_3\bm{g_3},v_1\bm{g_1,w}]
        +\lambda_3\lambda_2[u_3\bm{g_3},v_2\bm{g_2,w}]
        +\lambda_3\lambda_3[u_3\bm{g_3},v_3\bm{g_3,w}]\\
    \end{split}
\end{equation*}

上式中将$w$也展开同样是27项,三组求和则
$[u_i\bm{g_i},v_j\bm{g_j},v_k\bm{g_k}]$的系数是
$\lambda_i\lambda_j+\lambda_j\lambda_k+\lambda_k\lambda_i$
发现只有$i,j,k$不同时才非0,此时系数就是$I_2$,因此将剩余的系数都改成
$I_2$,求和(利用混合积线性)得到三组求和等于$I_2[\bm{u},\bm{v},\bm{w}]$

\subsubsection*{c}

设$\bm{C} = \bm{u}\bm{g_1}+\bm{v}\bm{g_2}+\bm{w}\bm{g_3}$

则容易根据行列式的定义$det(\bm{C})=\bm{[u,v,w]}$

$$
\bm{[B\cdot \bm{u},B\cdot \bm{v},B\cdot \bm{w}]}=det(\bm{B}\cdot \bm{C})
$$

根据2的结论,

$$
\bm{[B\cdot \bm{u},B\cdot \bm{v},B\cdot \bm{w}]}=det(\bm{B})det(\bm{C})
=I_3\bm{[u,v,w]}
$$

为所求。

\end{document}