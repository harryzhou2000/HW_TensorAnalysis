%!TEX program = xelatex
\documentclass[UTF8,zihao=5]{ctexart}


\title{{\bfseries 第1次作业}}
\author{周涵宇 2022310984}
\date{}

\usepackage[a4paper]{geometry}
\geometry{left=0.75in,right=0.75in,top=1in,bottom=1in}

\usepackage[
UseMSWordMultipleLineSpacing,
MSWordLineSpacingMultiple=1.5
]{zhlineskip}

\usepackage{fontspec}
\setmainfont{Times New Roman}
% \setmonofont{JetBrains Mono}
\setCJKmainfont{仿宋}[AutoFakeBold=true]
\setCJKsansfont{黑体}[AutoFakeBold=true]

% \usepackage{bm}

\usepackage{amsmath,amsfonts}
\usepackage{array}

\newcommand{\bm}[1]{{\mathbf{#1}}}

\newcommand{\trans}[0]{^\mathrm{T}}
\newcommand{\tran}[1]{#1^\mathrm{T}}
\newcommand{\hermi}[0]{^\mathrm{H}}

\newcommand*{\av}[1]{\left\langle{#1}\right\rangle}

\newcommand*{\avld}[1]{\frac{\overline{D}#1}{Dt}}

\newcommand*{\pd}[2]{\frac{\partial #1}{\partial #2}}

\newcommand*{\pdcd}[3]
{\frac{\partial^2 #1}{\partial #2 \partial #3}}


\begin{document}

\maketitle

\subsection*{1.1}


首先证明:
\begin{equation*}
    e^{ijk}e_{rst}=
    \left|
    \begin{matrix}
        \delta^i_r & \delta^i_s & \delta^i_t \\
        \delta^j_r & \delta^j_s & \delta^j_t \\
        \delta^k_r & \delta^k_s & \delta^k_t \\
    \end{matrix}
    \right|
    \equiv 
    \delta^i_r\delta^j_s\delta^k_t +
    \delta^i_s\delta^j_t\delta^k_r +
    \delta^i_t\delta^j_r\delta^k_s -
    \delta^i_t\delta^j_s\delta^k_r -
    \delta^i_s\delta^j_r\delta^k_t -
    \delta^i_r\delta^j_t\delta^k_s
\end{equation*}

考虑左侧为$\pm 1$的情况:
当$i,j,k$与$r,s,t$分别是$1,2,3$时,容易发现右侧是单位阵的行列式,等于左侧。
对$i,j,k$或者$r,s,t$循环移位时,由于相当于行列式交换两次行或者列,左右都不变。
对$i,j,k$或者$r,s,t$任意交换时,由于行列式交换一次行或者列,左右都变号。
因此所有非零的情况左右相等

考虑左侧为$0$:当$i,j,k$或者$r,s,t$一定有任意的重复,可知有两列或者两行重复,左右都是0。

总之,上述恒等式是成立的。
对上述恒等式$k,t$指标缩并,得到:
\begin{equation*}
    \begin{aligned}
        e^{ijk}e_{rsk}=&
    \delta^i_r\delta^j_s\delta^k_k +
    \delta^i_s\delta^j_k\delta^k_r +
    \delta^i_k\delta^j_r\delta^k_s -
    \delta^i_k\delta^j_s\delta^k_r -
    \delta^i_s\delta^j_r\delta^k_k -
    \delta^i_r\delta^j_k\delta^k_s\\
    =&
    3\delta^i_r\delta^j_s +
    \delta^i_s\delta^j_r +
    \delta^j_r\delta^i_s -
    \delta^j_s\delta^i_r -
    3\delta^i_s\delta^j_r -
    \delta^i_r\delta^j_s 
    \\
    =&\delta^i_r\delta^j_s -
    \delta^i_s\delta^j_r 
    \end{aligned}
\end{equation*}

进一步缩并可得:
\begin{equation*}
    \begin{aligned}
        e^{ijk}e_{rjk}
    =&\delta^i_r\delta^j_j -
    \delta^i_j\delta^j_r \\
    =&2\delta^i_r
    \end{aligned}
\end{equation*}

进一步缩并可得:
\begin{equation*}
    \begin{aligned}
        e^{ijk}e_{ijk}
    =&2\delta^i_i
    =6
    \end{aligned}
\end{equation*}

\subsection*{1.2}
在直角坐标展开,根据第一题,有:
\begin{equation*}
    \begin{aligned}
        \bm{u}\times(\bm{v}\times\bm{w})
        =&
        u^i\bm{e}_i\times(v^r\bm{e}_r\times w^s\bm{e}_s)\\
        =&
        u^i\bm{e}_i\times(v^r w^s e_{rsj}\bm{e}^j)\\
        =&
        u^i v^r w^s e_{rsj}e_{ijk}\bm{e}^k\\
        =&
        u^i v^r w^s (\delta^k_r\delta^i_s-\delta^k_s\delta^i_r)
        \bm{e}^k\\
        =&
        u^i v^k w^i \bm{e}^k- u^i v^i w^k\bm{e}^k\\
        =&
        (\bm{u}\cdot\bm{w})\bm{v}-(\bm{u}\cdot\bm{v})\bm{w}
    \end{aligned}
\end{equation*}

同样:
\begin{equation*}
    \begin{aligned}
        (\bm{u}\times\bm{v})\cdot(\bm{w}\times\bm{a})
        =&
        u^iv^je_{ijk}w^ra^se_{rsk}\\
        =&
        u^iv^jw^ra^s(\delta^i_r\delta^j_s -
        \delta^i_s\delta^j_r)\\
        =&
        u^iv^jw^ia^j-u^iv^jw^ja^i\\
        =&
        (\bm{u}\cdot\bm{w})(\bm{v}\cdot\bm{a})-
        (\bm{u}\cdot\bm{a})(\bm{v}\cdot\bm{w})
    \end{aligned}
\end{equation*}

\subsection*{1.3}
证明:

已知单位正交基下混合积定义的轮换性质
$$
[\bm{u},\bm{v},\bm{w}]=
(\bm{u}\times\bm{v})\bullet\bm{w}
=(\bm{v}\times\bm{w})\bullet\bm{u}
=(\bm{w}\times\bm{u})\bullet\bm{v}
$$

因此以下讨论只取轮换等价中的一个情况。

~\\

先讨论充分性

如果$\bm{u},\bm{v},\bm{w}$共面,则不妨设

$$
\bm{w}=a_1\bm{u}+a_2\bm{v}
$$

(上式及其轮换形式中必有一个成立,仅讨论此种)

则

$$
[\bm{u},\bm{v},\bm{w}]=
(\bm{u}\times\bm{v})\bullet\bm{w}
=(\bm{u}\times\bm{v})\bullet(a_1\bm{u}+a_2\bm{v})
=a_1(\bm{u}\times\bm{v})\bullet\bm{u}+
a_2(\bm{u}\times\bm{v})\bullet\bm{v}
$$

容易知道根据叉积的性质,上式最后两项都是0,则

$$
[\bm{u},\bm{v},\bm{w}]=0
$$

以上是充分性的逆否命题,因此得证。

~\\

再讨论必要性

已知$[\bm{u},\bm{v},\bm{w}]=0$

假设$\bm{u},\bm{v},\bm{w}$非共面,可知存在$\bm{w'}$

$$
\bm{w}=a_1\bm{u}+a_2\bm{v}+\bm{w'}
$$

且
$
\bm{w'}\neq 0, 
\bm{w'}\bullet\bm{u}=\bm{w'}\bullet\bm{u}=0
$。

即,必有$\bm{w}$在$span(\bm{u},\bm{v})$的正交空间
中的投影$\bm{w'}$非零。

又空间只有3维,$span(\bm{u}\times\bm{v})$
就是$span(\bm{u},\bm{v})$的正交空间
(子空间的正交空间的唯一性可以通过构造正交基
证明)

因此

$$
\left|[\bm{u},\bm{v},\bm{w}]\right|
=\left|\bm{w'}\bullet(\bm{u}\times\bm{v})\right|
=|\bm{w'}||\bm{u}\times\bm{v}|
$$

根据已知以上最后相乘两项都非0,
则知$[\bm{u},\bm{v},\bm{w}]\neq 0$,矛盾。

因此必要条件成立。

\subsection*{1.4}
取分量$\bm{x}=(1,1,1), \bm{y}=(0,0,0)$,代入即有:
$\bm{x}+\bm{y}=(1,1,1), \bm{y}+\bm{x}=(0,0,0)$,
则$\bm{y}+\bm{x}=2(\bm{x}+\bm{y})=(2,2,2)$,不构成阿贝尔群。

取分量$\bm{x}=(1,1,1)$,则$(1+1)\bm{x}=(2,1,2)\neq 1\bm{x} + 1\bm{x}=(2,2,2)$,
因此不是线性空间的数乘运算。

\subsection*{1.5}
以下讨论都是在$\mathbb{R}^3$中

\subsubsection*{i}

C1:共轭对称

此处实空间则退化为对称性,交换定义式右端的$x$和$y$发现与此前恒等,则证明。

C2:数乘线性

右端对$x$或者$y$而言都是一次项,无其他幂次,因此满足数乘线性。

C3:分配律

右端对$x$或者$y$而言都是一次项,每一项满足分配律,线性和后仍然满足。

C4:正定性

通过配方可以将其化为正定二次型
$$
(x_1+x_2)(y_1+y_2)+x_1y_1+x_2y_2+x_3y_3
$$

容易看到当$\bm{x}=\bm{y}$上式都是平方项,因此大于等于0且
向量为0的时候才是0。

或者采用更加
数值的方法:

$$
\bm{I'(x,y)}
=
[\bm{x}]\trans
\begin{bmatrix}
    2&1&0\\
    1&2&0\\
    0&0&1\\
\end{bmatrix}
[\bm{y}]=[\bm{x}]\trans
A
[\bm{y}]
$$

只需证明矩阵$A$的正定性,由于其严格对角占优且对角线为正则为正定。

或者,其特征值有$1,1,3$都是正数因此是正定的。

综上,满足内积的定义。

\subsubsection*{ii}

根据线性代数,所有的内积一定有对应双线性函数

\begin{equation}
\bm{I(x,y)}
=
[\bm{x}]\trans
M
[\bm{y}]
\end{equation}

且满足$M$是正定对称矩阵,而且根据i,容易证明只要有一个正定对称矩阵$M$依此定义出的
函数就是内积。

因此,所有三维内积算子的集合与所有三维正定对称矩阵的集合是一一映射的。

定义任意内积只需对其分量定义(1)形式的函数以及正定对称矩阵$M$即可。

如定义矩阵
$$
M=\begin{bmatrix}
    1&0&0\\
    0&2&0\\
    0&0&3\\
\end{bmatrix}
$$

则(1)式就是一个内积。

\subsection*{1.6}
根据直角坐标,通过系数矩阵求逆得到逆变基向量的分量:
\begin{equation*}
    \begin{aligned}
        \bm{g}^1=&\frac{1}{15}(5\bm{e}_1-5\bm{e}_2+5\bm{e}_3)\\
        \bm{g}^2=&\frac{1}{15}(-9\bm{e}_1+21\bm{e}_2-18\bm{e}_3)\\
        \bm{g}^3=&\frac{1}{15}(\bm{e}_1-4\bm{e}_2+7\bm{e}_3)\\
    \end{aligned}
\end{equation*}

逆变度量张量的分量则为:
\begin{equation*}
    g^{ij}=\bm{g}^i\cdot\bm{g}^j=\frac{1}{75}\begin{bmatrix}
        25 & -80 & 20\\
        -80 & 282 & -73 \\
        20 & -73 & 22
    \end{bmatrix}
\end{equation*}
% TODO



\subsection*{1.7}

当$\bm{u},\bm{v},\bm{w}$线性相关,左右都是0,显然成立,以下讨论线性无关情况。

记$\bm{u},\bm{v},\bm{w}=\bm{g}_1, \bm{g}_2, \bm{g}_3$
则一定有逆变基矢量$\bm{g}^1, \bm{g}^2, \bm{g}^3$,使得$\bm{g}^i\cdot\bm{g}_j=\delta^i_j$。
则讨论$\bm{a},\bm{b},\bm{c}$的逆变分量展开情况:
\begin{equation*}
    \begin{aligned}
        [\bm{a},\bm{b},\bm{c}]
        =&[a_i\bm{g}^i,b_j\bm{g}^j,c_k\bm{g}^k]\\
    \end{aligned}
\end{equation*}

考虑上式右端展开爱因斯坦求和有27项,其中非零的,只有$i,j,k$不等的情况,因此:
\begin{equation*}
    \begin{aligned}
        [\bm{a},\bm{b},\bm{c}]
        =&[a_i\bm{g}^i,b_j\bm{g}^j,c_k\bm{g}^k]\\
        =&(
            a_1b_2c_3 + a_2b_3c_1 + a_3b_1c_2
            -a_3b_2c_1 - a_2b_1c_3 - a_1b_3c_2
        )
        [\bm{g}^1, \bm{g}^2, \bm{g}^3]\\
        =&\left|
            \begin{matrix}
                a_1 &a_2 &a_3\\
                b_1 &b_2 &b_3\\
                c_1 &c_2 &c_3\\
            \end{matrix}
        \right|[\bm{g}^1, \bm{g}^2, \bm{g}^3]
    \end{aligned}
\end{equation*}

由于已知
\begin{equation*}
    \begin{aligned}
        \bm{g}^1=&
        \frac{1}{\sqrt{g}}(\bm{g}_2\times\bm{g}_3)\\
        \bm{g}^2=&
        \frac{1}{\sqrt{g}}(\bm{g}_3\times\bm{g}_1)\\
        \bm{g}^3=&
        \frac{1}{\sqrt{g}}(\bm{g}_1\times\bm{g}_2)\\
    \end{aligned}
\end{equation*}

因此,根据叉积公式(第二题):
\begin{equation*}
    \begin{aligned}
        [\bm{g}^1, \bm{g}^2, \bm{g}^3]=&
        \frac{1}{g^{\frac{3}{2}}}
        \left(
            (\bm{g}_2\times\bm{g}_3)\times
            (\bm{g}_3\times\bm{g}_1)
        \right)\cdot
        (\bm{g}_1\times\bm{g}_2)\\
        =&
        \frac{1}{g^{\frac{3}{2}}}
        \left[
            \left(
                (\bm{g}_2\times\bm{g}_3)\cdot\bm{g}_1
            \right)\bm{g}_3-
            \left(
                (\bm{g}_3\times\bm{g}_1)\cdot\bm{g}_3
            \right)\bm{g}_1
        \right]\cdot
        (\bm{g}_1\times\bm{g}_2)\\
        =&
        \frac{1}{g^{\frac{3}{2}}}
            [\bm{g}_1,\bm{g}_2,\bm{g}_3]\bm{g}_3\cdot
        (\bm{g}_1\times\bm{g}_2)\\
        =&
        \frac{1}{g^{\frac{3}{2}}}
            [\bm{g}_1,\bm{g}_2,\bm{g}_3][\bm{g}_1,\bm{g}_2,\bm{g}_3]\\
        =&
        \frac{1}{\sqrt{g}}
    \end{aligned}
\end{equation*}

因此,关于$[\bm{a},\bm{b},\bm{c}]$的展开可得:
\begin{equation*}
    \begin{aligned}
        [\bm{a},\bm{b},\bm{c}]
        =&\left|
            \begin{matrix}
                a_1 &a_2 &a_3\\
                b_1 &b_2 &b_3\\
                c_1 &c_2 &c_3\\
            \end{matrix}
        \right|\frac{1}{\sqrt{g}}\\
        \Rightarrow\ 
        [\bm{a},\bm{b},\bm{c}][\bm{g}_1,\bm{g}_2,\bm{g}_3]
        =&\left|
            \begin{matrix}
                a_1 &a_2 &a_3\\
                b_1 &b_2 &b_3\\
                c_1 &c_2 &c_3\\
            \end{matrix}
        \right|
    \end{aligned}
\end{equation*}

又根据逆变分量的计算方法:
\begin{equation*}
    \begin{aligned}
        a_i&=\bm{a}\cdot\bm{g}_i\\
        b_i&=\bm{b}\cdot\bm{g}_i\\
        c_i&=\bm{c}\cdot\bm{g}_i\\
    \end{aligned}
\end{equation*}

带入上式即为所求证公式。


\end{document}