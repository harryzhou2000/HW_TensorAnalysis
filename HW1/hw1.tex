%!TEX program = xelatex
\documentclass[UTF8,c5size]{ctexart}


\title{{\bfseries 第一次作业}}
\author{周涵宇 2018011600}
\date{}

\usepackage[a4paper]{geometry}
\geometry{left=0.75in,right=0.75in,top=1in,bottom=1in}

\usepackage[
UseMSWordMultipleLineSpacing,
MSWordLineSpacingMultiple=1.5
]{zhlineskip}

\usepackage{fontspec}
\setmainfont{Cambria Math}
% \setmonofont{JetBrains Mono}
\setCJKmainfont{仿宋}[AutoFakeBold=true]
\setCJKsansfont{黑体}[AutoFakeBold=true]

\usepackage{bm}
\usepackage{amsmath}
\usepackage{array}

\begin{document}

\maketitle

\subsection*{1}

证明:

首先明确,已知由于单位正交基下混合积定义的轮换性质

$$
[\bm{u},\bm{v},\bm{w}]=
(\bm{u}\times\bm{v})\bullet\bm{w}
=(\bm{v}\times\bm{w})\bullet\bm{u}
=(\bm{w}\times\bm{u})\bullet\bm{v}
$$

因此以下讨论只取轮换等价中的一个情况。

~\\

先讨论充分性

如果$\bm{u},\bm{v},\bm{w}$共面,则不妨设

$$
\bm{w}=a_1\bm{u}+a_2\bm{v}
$$

(上式及其轮换形式中必有一个成立,仅讨论此种)

则

$$
[\bm{u},\bm{v},\bm{w}]=
(\bm{u}\times\bm{v})\bullet\bm{w}
=(\bm{u}\times\bm{v})\bullet(a_1\bm{u}+a_2\bm{v})
=a_1(\bm{u}\times\bm{v})\bullet\bm{u}+
a_2(\bm{u}\times\bm{v})\bullet\bm{v}
$$

容易知道根据叉积的性质,上式最后两项都是0,则

$$
[\bm{u},\bm{v},\bm{w}]=0
$$

以上是充分性的逆否命题,因此得证。

~\\

再讨论必要性

已知$[\bm{u},\bm{v},\bm{w}]=0$

假设$\bm{u},\bm{v},\bm{w}$非共面,根据可知

$$
\bm{w}=a_1\bm{u}+a_2\bm{v}+\bm{w'}
$$

且
$
\bm{w'}\neq 0, 
\bm{w'}\bullet\bm{u}=\bm{w'}\bullet\bm{u}=0
$。

即,必有$\bm{w}$在$span(\bm{u},\bm{v})$的正交空间
中的投影$\bm{w'}$非零。

又空间只有3维,$span(\bm{u}\times\bm{v})$
就是$span(\bm{u},\bm{v})$的正交空间
(子空间的正交空间的唯一性可以通过构造正交基
证明)

因此

$$
\left|[\bm{u},\bm{v},\bm{w}]\right|
=\left|\bm{w'}\bullet(\bm{u}\times\bm{v})\right|
=|\bm{w'}||\bm{u}\times\bm{v}|
$$

根据已知以上最后相乘两项都非0,
则知$[\bm{u},\bm{v},\bm{w}]\neq 0$。

因此必要条件成立。

\subsection*{2}

不妨取三维标准正交基$\bm{e_i}$讨论分量,给出

\begin{equation*}
    \begin{split}
        \bm{u}\times(\bm{v}\times\bm{w})&=
        e_{rsi}e^{ijk}u^sv_jw_k\bm{e^r}\\
        &=
        \left|
        \begin{matrix}
            \delta^i_r&\delta^i_s&\delta^i_i\\
            \delta^j_r&\delta^j_s&\delta^j_i\\
            \delta^k_r&\delta^k_s&\delta^k_i\\
        \end{matrix}
        \right|u^sv_jw_k\bm{e^r}\\
        &=
        (\delta^i_r\delta^j_s\delta^k_i-\delta^i_r\delta^j_i\delta^k_s
        +\delta^i_s\delta^j_i\delta^k_r-\delta^i_s\delta^j_r\delta^k_i
        +\delta^i_i\delta^j_r\delta^k_s-\delta^i_i\delta^j_s\delta^k_r)
        u^sv_jw_k\bm{e^r}\\
        &=
        (\delta^k_r\delta^j_s-\delta^j_r\delta^k_s
        +\delta^j_s\delta^k_r-\delta^k_s\delta^j_r
        +3\delta^j_r\delta^k_s-3\delta^j_s\delta^k_r)
        u^sv_jw_k\bm{e^r}\\
        &=
        (\delta^j_r\delta^k_s-\delta^j_s\delta^k_r)
        u^sv_jw_k\bm{e^r}\\
        &=u^kv_rw_k\bm{e^r}-u^jv_jw_r\bm{e^r}\\
        &=(\bm{u}\bullet\bm{w})\bm{v}-(\bm{u}\bullet\bm{v})\bm{w}
    \end{split}
\end{equation*}

\subsection*{3}

不妨取三维标准正交基$\bm{e_i}$讨论分量,给出

\begin{equation*}
    \begin{split}
        (\bm{a}\times\bm{b})\bullet(\bm{c}\times\bm{d})&=
        e^{ijk}e_{ist}a_jb_kc^sd^t\\
        &=(\delta^j_s\delta^k_t-\delta^j_t\delta^k_s)a_jb_kc^sd^t\\
        &=a_sb_tc^sd^t-a_tb_sc^sd^t\\
        &=(\bm{a}\bullet\bm{c})(\bm{b}\bullet\bm{d})-
        (\bm{a}\bullet\bm{b})(\bm{c}\bullet\bm{d})
    \end{split}
\end{equation*}

\subsection*{4}

不妨记$\bm{a},\bm{b},\bm{c}=\bm{a_{(1)}},\bm{a_{(2)}},\bm{a_{(3)}}$,  \ \ 
$\bm{u},\bm{v},\bm{w}=\bm{u_{(1)}},\bm{u_{(2)}},\bm{u_{(3)}}$

则已知在正交标准系下,可知

\begin{equation*}
    \begin{split}
        \text{左侧}&=det([a_{(i)k}])det([u_{(j)k}])\\
        &=det([a_{(i)k}][u_{(j)k}])=det([a_{(i)k}u_{(j)k}])\\
        &=det([\bm{a_{(i)}}\bullet\bm{u_{(j)}}])\\
        &=\text{右侧}
    \end{split}
\end{equation*}

以上推导应用了矩阵乘法和其行列式运算的交换性,以及混合积的行列式写法。

\subsection*{5}

加法:

$$
\bm{x}+\bm{y}=(x_1+2y_1,x_2+2y_2,x_3+2y_3)
$$

则考虑:

\begin{equation*}
    \begin{split}
        (\bm{x}+\bm{y})+\bm{z}&=(x_1+2y_1+2z_1,x_2+2y_2+2z_2,x_3+2y_3+2z_3)\\
        \bm{x}+(\bm{y}+\bm{z})&=(x_1+2y_1+4z_1,x_2+2y_2+4z_2,x_3+2y_3+4z_3)
    \end{split}
\end{equation*}

不满足分配律,不构成群。

数乘:

\begin{equation*}
    \alpha\bm{x}=(\alpha x_1, x_2, \alpha x_3)
\end{equation*}

若$x_2\neq 0$且$\alpha= 0$则$\bm{x}\neq 0\text{(对于一般的加法定义的0矢量)}$,
因此不满足数乘的归零要求。


\subsection*{6}

设图中$A,B,C$点全部在单位球上。

根据几何考虑$\angle A$的表达:

\begin{equation*}
    \begin{split}
        \cos{A}=\bm{n_{AOC}}\bullet\bm{n_{AOB}}=
        \frac{\bm{OB}\times\bm{OA}}{\left\|\bm{OB}\times\bm{OA}\right\|}
        \bullet
        \frac{\bm{OC}\times\bm{OA}}{\left\|\bm{OC}\times\bm{OA}\right\|}
        =
        \frac{\bm{OB}\times\bm{OA}}{\sin{\gamma}}
        \bullet
        \frac{\bm{OC}\times\bm{OA}}{\sin{\beta}}
    \end{split}
\end{equation*}

则有:

\begin{equation*}
    \begin{split}
        \text{右侧}&=(\bm{OA}\bullet\bm{OB})(\bm{OA}\bullet\bm{OC})+\sin{\gamma}\frac{\bm{OB}\times\bm{OA}}{\sin{\gamma}}
        \bullet
        \sin{\beta}\frac{\bm{OC}\times\bm{OA}}{\sin{\beta}}
        \\&=(\bm{OA}\bullet\bm{OB})(\bm{OA}\bullet\bm{OC})+
        (\bm{OB}\times\bm{OA})
        \bullet
        (\bm{OC}\times\bm{OA})
        \\&=(\bm{OA}\bullet\bm{OB})(\bm{OA}\bullet\bm{OC})+
        (\bm{OB}\bullet\bm{OC})(\bm{OA}\bullet\bm{OA})-(\bm{OA}\bullet\bm{OB})
        \\&=\bm{OB}\bullet\bm{OC}=\cos{\alpha}=\text{右侧}
    \end{split}
\end{equation*}

\end{document}