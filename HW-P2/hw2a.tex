%!TEX program = xelatex
\documentclass[UTF8,zihao=5]{ctexart}


\title{{\bfseries 第2次作业}}
\author{周涵宇 2022310984}
\date{}

\usepackage[a4paper]{geometry}
\geometry{left=0.75in,right=0.75in,top=1in,bottom=1in}

\usepackage[
UseMSWordMultipleLineSpacing,
MSWordLineSpacingMultiple=1.5
]{zhlineskip}

\usepackage{fontspec}
\setmainfont{Times New Roman}
% \setmonofont{JetBrains Mono}
\setCJKmainfont{仿宋}[AutoFakeBold=true]
\setCJKsansfont{黑体}[AutoFakeBold=true]

% \usepackage{bm}

\usepackage{amsmath,amsfonts}
\usepackage{array}

\newcommand{\bm}[1]{{\mathbf{#1}}}

\newcommand{\trans}[0]{^\mathrm{T}}
\newcommand{\tran}[1]{#1^\mathrm{T}}
\newcommand{\hermi}[0]{^\mathrm{H}}
\newcommand{\conj}[1]{\overline{#1}}

\newcommand*{\av}[1]{\left\langle{#1}\right\rangle}

\newcommand*{\avld}[1]{\frac{\overline{D}#1}{Dt}}

\newcommand*{\pd}[2]{\frac{\partial #1}{\partial #2}}

\newcommand*{\pdcd}[3]
{\frac{\partial^2 #1}{\partial #2 \partial #3}}


\begin{document}

\maketitle

\subsection*{2.1}

不妨在笛卡尔坐标下展开为分量,则:

\begin{equation}
    \begin{aligned}
        \bm{A}\cdot\bm{W}= & A_{ij}B^j_{\cdot k}\bm{e}^i\bm{e}^k
        =\varepsilon_{ijr}a^r\varepsilon^j_{\cdot ks}w^s\bm{e}^i\bm{e}^k
        =\varepsilon_{rij}\varepsilon^{\cdot\cdot j}_{ks}a^rw^s\bm{e}^i\bm{e}^k
        =(\delta_{rk}\delta_{is} - \delta_{rs}\delta_{ik})a^rw^s\bm{e}^i\bm{e}^k \\
        =                  & a^kw^i\bm{e}^i\bm{e}^k-a^sw^s\bm{e}^k\bm{e}^k
        =\bm{w}\bm{a}-(\bm{w}\cdot\bm{a})\bm{I}
    \end{aligned}
\end{equation}

因此,当$\bm{W}=\bm{A}$,$\bm{A}^2=\bm{a}\bm{a}-(\bm{a}\cdot\bm{a})\bm{I}$。

由于$\bm{A}$是反对称的,
$tr(\bm{A}\cdot\bm{A}) = -tr(\bm{A}\cdot\bm{A}\trans)=-\bm{A}:\bm{A} $,
又$tr(\bm{A}\cdot\bm{A})=tr(\bm{a}\bm{a}-(\bm{a}\cdot\bm{a})\bm{I})=\bm{a}\cdot\bm{a}-3\bm{a}\cdot\bm{a}=-2\bm{a}\cdot\bm{a}$
可知$\bm{A}:\bm{A}=2\bm{a}\cdot\bm{a}$。

\subsection*{2.2}
\subsubsection*{(i)}
可以通过Caylay-Hamilton定理给出,三维二阶张量的3次幂可以由更低次幂表示,
左右同乘本张量递推即有任意高于或等于3次的幂可以由0,1,2次幂表示。



对这样的等式两端取迹,则可知:
\begin{equation}
    J_{3+k}-I_1J_{2+k}+I_2J_{1+k}-I_3J_{0+k}=0,\ \  k=0,1,2..., J_0=3
\end{equation}

任意高于或等于3次的$J$可以被三个不变量以及$J_1,J_2$表示,同时可以证明$J_1,J_2$
完全可以由三个不变量表示,因此,所有的$J$共有三个自由度,可以完全由三个不变量表示。

\subsubsection*{(ii)}
对于三维二阶张量$\bm{T}$,主迹数如下:

\begin{equation}
    \begin{aligned}
        J_1 & =T^i_{\cdot i},\
        J_2=T^i_{\cdot j}T^j_{\cdot i},\
        J_3=T^i_{\cdot j}T^j_{\cdot k}T^k_{\cdot i} \\
            & \text{(本式中$J,I$的上标都是幂)}
    \end{aligned}
\end{equation}

根据特征方程已经得出:

\begin{equation}
    \begin{aligned}
        I_1= & T^i_{\cdot i}=J_1                                            \\
        I_2= & \frac{1}{2}\left(T^i_{\cdot i}T^j_{\cdot j}
        -T^i_{\cdot j}T^j_{\cdot i}\right)
        =\frac{1}{2}(J_1^2-J_2)                                             \\
        I_3= & \det(\bm{T})=e_{rst}T^r_{\cdot 1}T^s_{\cdot 2}T^t_{\cdot 3}=
        \frac{1}{6}e^{ijk}e_{rst}T^r_{\cdot i}T^s_{\cdot j}T^t_{\cdot k}    \\
        =    &
        \frac{1}{6}
        \left(\delta^i_r\delta^j_s\delta^k_t +
        \delta^i_s\delta^j_t\delta^k_r +
        \delta^i_t\delta^j_r\delta^k_s -
        \delta^i_t\delta^j_s\delta^k_r -
        \delta^i_s\delta^j_r\delta^k_t -
        \delta^i_r\delta^j_t\delta^k_s\right)
        T^r_{\cdot i}T^s_{\cdot j}T^t_{\cdot k}                             \\
        =    &
        \frac{1}{6}
        \left(
        T^i_{\cdot i}T^j_{\cdot j}T^k_{\cdot k}+
        T^r_{\cdot s}T^s_{\cdot t}T^t_{\cdot r}+
        T^r_{\cdot t}T^s_{\cdot r}T^t_{\cdot s}-
        T^r_{\cdot t}T^s_{\cdot s}T^t_{\cdot r}-
        T^r_{\cdot s}T^s_{\cdot r}T^t_{\cdot t}-
        T^r_{\cdot r}T^s_{\cdot t}T^t_{\cdot s}
        \right)                                                             \\
        =    &
        \frac{1}{6}
        \left(
        J_1^3+2J_3-3J_1J_2
        \right)                                                             \\
             & \text{(本式中$J,I$的上标都是幂)}
    \end{aligned}
\end{equation}

\subsubsection*{(iii)}

根据上一个问题,求解(三角的)非线性方程组即可,通过代入法直接解得:
\begin{equation}
    \begin{aligned}
        J_1= & I_1                                    \\
        J_2= & I_1^2-2I_2                             \\
        J_3= & 3I_3+\frac{3J_1J_2}{2}-\frac{J_1^3}{2}
        =3I_3+\frac{3I_1(I_1^2-2I_2)}{2}-\frac{I_1^3}{2}
        =I_1^3-3I_1I_2+3I_3                           \\
             & \text{(本式中$J,I$的上标都是幂)}
    \end{aligned}
\end{equation}

\subsection*{2.3}
\subsubsection*{(i)}
充要条件的表述不唯一,可以是“其特征值有且只有一个非0”,
也可以是“将一组基映射后张成一维空间”,
或者“一组基映射后成为共线的向量组且至少一个非零”。

容易证明这三个表述等价,现给出最后一个的证明。
由于当$\bm{a},\bm{b}$至少一个非零时,两侧乘除一个非零标量,
不影响等式,此处的唯一性应当指的是在共线的意义下所有的非零向量
算作一个。

首先指出,定理进一步约束为$\bm{a},\bm{b},\bm{B}$都是非0的,
否则$B$是0,则其中一个为0另一个任取都满足,
是非唯一的。

必要性:已知有唯一的$\bm{a},\bm{b}\neq\bm{0}\ \ \ \ s.t.\ \ \  \bm{B}=\bm{a}\bm{b}$,
则对于基向量$\bm{g}_i$,映射后为$\bm{a}(\bm{b}\cdot \bm{g}_i)$,新的向量组一定在$\bm{a}$张成的空间
中,且由于$\bm{b}$非0,其协变分量不都是0,则条件满足,因此是必要条件。

充分性:已知条件:$\bm{B}\cdot\bm{g}_i$共线,则取此方向的一个非零向量$\bm{a}$,可给出
$\bm{B}\cdot\bm{g}_i=b_i\bm{a}$,其系数$b_i$不都是0,因此存在非零向量$\bm{b}=b_i\bm{g}^i$,
使得$\bm{B}\cdot\bm{g}_i=\bm{a}\bm{b}\cdot\bm{g}_i$成立,由于对基向量成立,对所有向量都成立,
因此$\bm{B}=\bm{a}\bm{b}$,上述过程确定的$\bm{a}, \bm{b}$各自只能在确定的直线上,共线意义是唯一的。

\subsubsection*{(ii)}

三维情况。
对于任意的二阶张量,其表示为$\bm{B}=B^{ij}\bm{g}_i\bm{g}_j$,取$K=9$,则容易得到一组
$\bm{a}_i,\bm{b}_i, i=1,2,\dots,9$满足关系,且由于基向量的任意性取法不唯一。现证明如果正则,
最少表示为3个。

正则指的是三维空间映射为三维空间,也就是一组基映射后线性无关,还是一组基。
则如果存在$\bm{B}=\sum_{i=1}^{3}{\bm{a}_i\bm{b}_i}$,则映射$\bm{g}_i$到
$\bm{B}\cdot\bm{g}_i=\sum_{i=1}^{K}{\bm{a}_i\left(\bm{b}_i\cdot\bm{g}_i\right)}$
如果$K<3$,发现映射后三个向量都是$\bm{a}_1,\bm{a}_2$表示的,一定线性相关,因此不是正则的,因此有
$K\ge3$。

\subsection*{2.4}


\subsubsection*{(i)}
\begin{equation}
    \begin{aligned}
        \frac{
        [\bm{B}\cdot\bm{u}, \bm{v}, \bm{w}]+
        [\bm{u}, \bm{B}\cdot\bm{v}, \bm{w}]+
        [\bm{u}, \bm{v}, \bm{B}\cdot\bm{w}]}{[\bm{u}, \bm{v}, \bm{w}]}
        = &
        \frac{
        B^i_{\cdot m}u^mv^jw^k\epsilon_{ijk}+
        u^iB^j_{\cdot m}v^mw^k\epsilon_{ijk}+
        u^iv^jB^k_{\cdot m}w^m\epsilon_{ijk}}
        {[\bm{u}, \bm{v}, \bm{w}]} \\
        = &
        \frac{
        B^i_{\cdot m}u^mv^jw^k\epsilon_{ijk}+
        u^kB^i_{\cdot m}v^mw^j\epsilon_{ijk}+
        u^jv^kB^i_{\cdot m}w^m\epsilon_{ijk}}
        {[\bm{u}, \bm{v}, \bm{w}]} \\
    \end{aligned}
\end{equation}

观察上式的分子,当$i=1,m=2$,
分子为$B^1_{\cdot 2}u^2v^jw^k\epsilon_{1jk}+B^1_{\cdot 2}u^kv^2w^j\epsilon_{1jk}+B^1_{\cdot 2}u^jv^kw^2\epsilon_{1jk}$,求和,
只有$j=2,k=3$或者$j=3,k=2$的部分是非零,发现第一项在$j=2,k=3$时与
第二项在$j=3,k=2$时刚好相反,同理六项中三对都为符号相反、位置依序,因此这些项求和是0。

求和结果:
\begin{equation}
    \begin{aligned}
        \frac{
        [\bm{B}\cdot\bm{u}, \bm{v}, \bm{w}]+
        [\bm{u}, \bm{B}\cdot\bm{v}, \bm{w}]+
        [\bm{u}, \bm{v}, \bm{B}\cdot\bm{w}]}{[\bm{u}, \bm{v}, \bm{w}]}
        = &
        \frac{
        B^1_{\cdot 1}u^1v^jw^k\epsilon_{1jk}+
        B^1_{\cdot 1}u^kv^1w^j\epsilon_{1jk}+
        B^1_{\cdot 1}u^jv^kw^1\epsilon_{1jk}}
        {[\bm{u}, \bm{v}, \bm{w}]} \\
        + &
        \frac{
        B^2_{\cdot 2}u^2v^jw^k\epsilon_{2jk}+
        B^2_{\cdot 2}u^kv^2w^j\epsilon_{2jk}+
        B^2_{\cdot 2}u^jv^kw^2\epsilon_{2jk}}
        {[\bm{u}, \bm{v}, \bm{w}]} \\
        + &
        \frac{
        B^3_{\cdot 3}u^3v^jw^k\epsilon_{3jk}+
        B^3_{\cdot 3}u^kv^3w^j\epsilon_{3jk}+
        B^3_{\cdot 3}u^jv^kw^3\epsilon_{3jk}}
        {[\bm{u}, \bm{v}, \bm{w}]} \\
    \end{aligned}
\end{equation}

根据行列编号,上式分子表示为第$(m,n)$项,
9项都为两个子项的
求和,符号相反。
为第$m$行第$n$列的
一项。
考虑上式中含有$u^1v^2w^3$的子项中,为
$(1,1)$取$j=2,k=3$,
$(2,2)$取$j=3,k=1$,
$(3,3)$取$j=1,k=2$,
求和为$B^m_{\cdot m}u^1v^2w^3$;
含有$u^3v^1w^2$的项为$(1,2)$,$(2,3)$,$(3,1)$的子项,
求和为
$B^m_{\cdot m}u^3v^1w^2$;
含有$u^2v^3w^1$的项为$(1,3)$,$(2,1)$,$(3,2)$的子项,
求和为
$B^m_{\cdot m}u^1v^2w^3$。
,这样所有含有正号的子项都归纳完毕。
含有负号的子项相当于$u,v,w$上标逆序,其选取的分子的位置
也是第二个坐标恰好逆序。可以发现,其选取的$j,k$都与正号时相反,
因此18个子项都归纳为6项:

\begin{equation}
    \begin{aligned}
        \frac{
        [\bm{B}\cdot\bm{u}, \bm{v}, \bm{w}]+
        [\bm{u}, \bm{B}\cdot\bm{v}, \bm{w}]+
        [\bm{u}, \bm{v}, \bm{B}\cdot\bm{w}]}{[\bm{u}, \bm{v}, \bm{w}]}
        = &
        \frac{
        B^m_{\cdot m}
        \left(
        u^1v^2w^3+u^2v^3w^1+u^3v^1w^2-
        u^1v^3w^2-u^2v^1w^3-u^3v^2w^1
        \right)}
        {[\bm{u}, \bm{v}, \bm{w}]} \\
        = &
        \frac{
        B^m_{\cdot m}
        u^iv^jw^k\epsilon_{ijk}}
        {[\bm{u}, \bm{v}, \bm{w}]} \\
        = &
        \frac{
        B^m_{\cdot m}
        [\bm{u}, \bm{v}, \bm{w}]}
        {[\bm{u}, \bm{v}, \bm{w}]} \\
        = &
        B^m_{\cdot m}=I_1
    \end{aligned}
\end{equation}

\subsubsection*{(ii)}

根据结论:

\begin{equation}
    \begin{aligned}
        [\bm{h}_1,\bm{h}_2,\bm{h}_3][\bm{g}_1,\bm{g}_2,\bm{g}_3]
        = & \left|
        \begin{matrix}
            \bm{h}_1\cdot\bm{g}_1 & \bm{h}_1\cdot\bm{g}_2 & \bm{h}_1\cdot\bm{g}_3 \\
            \bm{h}_2\cdot\bm{g}_1 & \bm{h}_2\cdot\bm{g}_2 & \bm{h}_2\cdot\bm{g}_3 \\
            \bm{h}_3\cdot\bm{g}_1 & \bm{h}_3\cdot\bm{g}_2 & \bm{h}_3\cdot\bm{g}_3 \\
        \end{matrix}
        \right|
    \end{aligned}
\end{equation}

选取$\{\bm{u},\bm{v},\bm{w}\}\equiv\left\{\bm{h}_i\right\}$的逆变基$\bm{h}^i$
使得$\bm{h}_i\cdot\bm{h}^j=\delta^i_j$,上下乘$[\bm{h}^1,\bm{h}^2,\bm{h}^3]$,
则有:

\begin{equation}
    \begin{aligned}
          & \frac{
        [\bm{B}\cdot\bm{u}, \bm{B}\cdot\bm{v}, \bm{w}]+
        [\bm{u}, \bm{B}\cdot\bm{v}, \bm{B}\cdot\bm{w}]+
        [\bm{B}\cdot\bm{u}, \bm{v}, \bm{B}\cdot\bm{w}]}{[\bm{u}, \bm{v}, \bm{w}]}
        \\
        = &
        \frac{
            \left|
            \begin{matrix}
                \bm{h}^1\cdot\bm{B}\cdot\bm{h}_1 & \bm{h}^1\cdot\bm{B}\cdot\bm{h}_2 \\
                \bm{h}^2\cdot\bm{B}\cdot\bm{h}_1 & \bm{h}^2\cdot\bm{B}\cdot\bm{h}_2 \\
            \end{matrix}
            \right|
            +
            \left|
            \begin{matrix}
                \bm{h}^2\cdot\bm{B}\cdot\bm{h}_2 & \bm{h}^2\cdot\bm{B}\cdot\bm{h}_3 \\
                \bm{h}^3\cdot\bm{B}\cdot\bm{h}_2 & \bm{h}^3\cdot\bm{B}\cdot\bm{h}_3 \\
            \end{matrix}
            \right|
            +
            \left|
            \begin{matrix}
                \bm{h}^3\cdot\bm{B}\cdot\bm{h}_3 & \bm{h}^3\cdot\bm{B}\cdot\bm{h}_1 \\
                \bm{h}^1\cdot\bm{B}\cdot\bm{h}_3 & \bm{h}^1\cdot\bm{B}\cdot\bm{h}_1 \\
            \end{matrix}
            \right|
        }{\det{[\delta_i^j]}} \\
        = &
        \left|
        \begin{matrix}
            B^1_{\cdot1} & B^1_{\cdot2} \\
            B^2_{\cdot1} & B^2_{\cdot2} \\
        \end{matrix}
        \right|
        +
        \left|
        \begin{matrix}
            B^2_{\cdot2} & B^2_{\cdot3} \\
            B^3_{\cdot2} & B^3_{\cdot3} \\
        \end{matrix}
        \right|
        +
        \left|
        \begin{matrix}
            B^3_{\cdot3} & B^3_{\cdot1} \\
            B^1_{\cdot3} & B^1_{\cdot1} \\
        \end{matrix}
        \right|
        =\frac{1}{2}\left(B^i_{\cdot i}B^j_{\cdot j}-B^i_{\cdot j}B^j_{\cdot i}\right)
    \end{aligned}
\end{equation}

上式分量在协变基是$\bm{u},\bm{v},\bm{w}$的混变分量。
上式进行坐标变换,代入坐标变换系数后,在直角坐标下形式是不变的,因此为$I_2$。
或者可以理解为,其表示为主迹数后,由于主迹数写作混变分量缩并时,
在任意坐标下的形式一样,上述形式即为
$I_2$。

\subsubsection*{(iii)}

与上一小问相同选取$\{\bm{u},\bm{v},\bm{w}\}\equiv\left\{\bm{h}_i\right\}$
为协变基,并上下同乘逆变基的混合积:

\begin{equation}
    \begin{aligned}
          & \frac{
            [\bm{B}\cdot\bm{u}, \bm{B}\cdot\bm{v}, \bm{B}\bm{w}]}
        {[\bm{u}, \bm{v}, \bm{w}]}
        \\
        = &
        \frac{
            \left|
            \begin{matrix}
                \bm{h}^1\cdot\bm{B}\cdot\bm{h}_1 & \bm{h}^1\cdot\bm{B}\cdot\bm{h}_2 & \bm{h}^1\cdot\bm{B}\cdot\bm{h}_3 \\
                \bm{h}^2\cdot\bm{B}\cdot\bm{h}_1 & \bm{h}^2\cdot\bm{B}\cdot\bm{h}_2 & \bm{h}^2\cdot\bm{B}\cdot\bm{h}_3 \\
                \bm{h}^3\cdot\bm{B}\cdot\bm{h}_1 & \bm{h}^3\cdot\bm{B}\cdot\bm{h}_2 & \bm{h}^3\cdot\bm{B}\cdot\bm{h}_3 \\
            \end{matrix}
            \right|
        }{\det{[\delta_i^j]}} \\
        = &
        \left|
        \begin{matrix}
            B^1_{\cdot1} & B^1_{\cdot2} & B^1_{\cdot3} \\
            B^2_{\cdot1} & B^2_{\cdot2} & B^2_{\cdot3} \\
            B^3_{\cdot1} & B^3_{\cdot2} & B^3_{\cdot3} \\
        \end{matrix}
        \right| = \det{(\bm{B})}=I_3
    \end{aligned}
\end{equation}

上式倒数第二个等式,是由于混变分量展开后写成广义Kronecker $\delta$之后,坐标变换到直角坐标时,
作为混变分量形式不改变,因此混变分量的行列式就是张量的行列式。

\subsection*{2.5}

容易证明,二阶反对称张量的特征根的实部是0 %(因为用H而不是T的内积才正定)
;根据代数基本定理,则有非0根两两为的为共轭虚根。

设$\bm{A}$有两个不同且不共轭的根,$si, ui, s\neq\pm u$,对应特征向量为$\bm{v}, \bm{w}$,
则有:$\bm{A}\cdot\bm{v}=(si)\bm{v}, \bm{A}\cdot\bm{w}=(ui)\bm{w}$,
反对称:$-\conj{\bm{v}}\cdot\bm{A}=(si)\conj{\bm{v}}$;
因此:$-\conj{\bm{v}}\cdot(ui)\bm{w}=-\conj{\bm{v}}\cdot\bm{A}\cdot\bm{w}=(si)\conj{\bm{v}}\cdot\bm{w}$;
因此:$(ui)\conj{\bm{v}}\cdot\bm{w}=(-si)\conj{\bm{v}}\cdot\bm{w}$;
由于$u\neq-s$,上式有$\conj{\bm{v}}\cdot\bm{w}=0$,即为垂直。

(本题中点积符号代表的是将协变、逆变分量直接缩并,右侧没有取共轭)

\subsection*{2.6}

线性空间中二阶张量$\bm{A}$(即线性变换),混变分量一定在一组基(或者说相似变换)下成为Jordan标准型,
写作矩阵形式(方便下面讨论):
$[\bm{A}]=[\bm{T}][\bm{J}][\bm{T}]^{-1}$,其中$[\bm{T}]$坐标转换系数矩阵。
其中Jordan标准型为分块对角阵,其中主对角线为特征值,上1对角线为0或1,连续的1构成一个
Jordan链,每个Jordan链对应的对角块矩阵一定是同一个特征值构成的,即有设$m\times m$矩阵
为$[j_m(\lambda)]$称作Jordan块,其主对角线是$\lambda$,上1对角线全为1:
\begin{equation}
    [j_m(\lambda)]=\begin{bmatrix}
        \lambda & 1       & 0      & \cdots  & 0       \\
        0       & \lambda & 1      & \cdots  & 0       \\
        0       & 0       & \ddots & \ddots  & 0       \\
        0       & \cdots  & 0      & \lambda & 1       \\
        0       & \cdots  & 0      & 0       & \lambda \\
    \end{bmatrix}_{m\times m}
\end{equation}
Jordan块大小为1时才是对角阵。

总有:$[\bm{J}]=diag([j_{m1}(\lambda_{i1})],[j_{m2}(\lambda_{i2})],\dots,[j_{mq}(\lambda_{iq})])$
,为$q$个Jordan块组成。其中$\lambda_{ia},\lambda_{ib}$可能相等,将其排序,则每个重特征根构成对角阵,
是若干Jordan块对角排列形成。也就是说Jordan块的大小一定不大于其代数重数。

以上Jordan分解定理此处不证明。

下面证明Hamilton-Caylay定理,即$f([\bm{A}])=0, f(\lambda)=\det(\lambda[\bm{I}]-[\bm{A}])$。

设有不同特征根$\lambda_1,\lambda_2,\dots\lambda_k$,其代数重数分别为$r_1,r_2,...,r_k$,
因此$f(\lambda)=\prod_{i=1}^k{(\lambda-\lambda_i)^{r_i}}$,则:

\begin{equation}
    \begin{aligned}
        f([\bm{A}])=\prod_{i=1}^k{([\bm{A}]-\lambda_i[\bm{I}])^{r_i}}
        =\prod_{i=1}^k{\left[[\bm{T}]([\bm{J}]-\lambda_i[\bm{I}])[\bm{T}]^{-1}\right]^{r_i}}
            =[\bm{T}]\left[\prod_{i=1}^k{([\bm{J}]-\lambda_i[\bm{I}])^{r_i}}\right][\bm{T}]^{-1}
    \end{aligned}
\end{equation}

考虑上式乘积中的$([\bm{J}]-\lambda_i[\bm{I}])^{r_i}$,其中$\lambda_i$对应的Jordan块大小不大于$r_i$。
由于:

$$
    [j_m(\lambda_i)]-\lambda_i[\bm{I}_m]=\begin{bmatrix}
        0 & 1      & 0      & \cdots & 0 \\
        0 & 0      & 1      & \cdots & 0 \\
        0 & 0      & \ddots & \ddots & 0 \\
        0 & \cdots & 0      & 0      & 1 \\
        0 & \cdots & 0      & 0      & 0 \\
    \end{bmatrix}_{m\times m}
$$

而
$$
    \begin{bmatrix}
        0 & 1      & 0      & \cdots & 0 \\
        0 & 0      & 1      & \cdots & 0 \\
        0 & 0      & \ddots & \ddots & 0 \\
        0 & \cdots & 0      & 0      & 1 \\
        0 & \cdots & 0      & 0      & 0 \\
    \end{bmatrix}_{m\times m}^2=\begin{bmatrix}
        0 & 0      & 1      & \cdots & 0 \\
        0 & 0      & 0      & \cdots & 0 \\
        0 & 0      & \ddots & \ddots & 1 \\
        0 & \cdots & 0      & 0      & 0 \\
        0 & \cdots & 0      & 0      & 0 \\
    \end{bmatrix}_{m\times m}
$$

即每自乘一次,上述矩阵1的位置上移1,因此$\left[[j_m(\lambda_i)]-\lambda_i[\bm{I}_m]\right]^m=[\bm{0}_m]$。
由于$([\bm{J}]-\lambda_i[\bm{I}])^{r_i}$中每个对角块的幂次都不小于其大小,其必为0,
因此这个矩阵在$\lambda_1$对应的位置都是0,因此对每个特征值给出的矩阵乘积后,全部为0。

即有Hamilton-Caylay定理。考虑坐标变换后对张量实体同样成立。

对Hamilton-Caylay定理两端乘$\bm{A}$的$0,1,2...$次幂后,
可知任意高次的幂都可以用$0,1,2,...n-1$次幂表示。
具体而言,

\begin{equation}
    \begin{aligned}
        \bm{A}^{n+k}+\sum_{i=1}^n{(-1)^nI_n\bm{A}^{n-i+k}}=0,\ \ k=0,1,2,...
    \end{aligned}
\end{equation}

则有递推公式:

\begin{equation}
    \begin{aligned}
        \bm{A}^{n+k}=-\sum_{i=1}^n{(-1)^nI_n\bm{A}^{n-i+k}},\ \ k=0,1,2,...
    \end{aligned}
\end{equation}

给出$\bm{A}^0,\bm{A}^1,\dots,\bm{A}^{n-1}$后,任意次幂都可以由这些初值表示,即为$n$维表现定理。

通过递推关系和一些级数的技巧可以推出三维情况下,任意大于2次幂时的表示系数的显式表达,具体见郑老师论文。




% !delete below
% \

% \

% \


% 首先证明:
% \begin{equation*}
%     e^{ijk}e_{rst}=
%     \left|
%     \begin{matrix}
%         \delta^i_r & \delta^i_s & \delta^i_t \\
%         \delta^j_r & \delta^j_s & \delta^j_t \\
%         \delta^k_r & \delta^k_s & \delta^k_t \\
%     \end{matrix}
%     \right|
%     \equiv
%     \delta^i_r\delta^j_s\delta^k_t +
%     \delta^i_s\delta^j_t\delta^k_r +
%     \delta^i_t\delta^j_r\delta^k_s -
%     \delta^i_t\delta^j_s\delta^k_r -
%     \delta^i_s\delta^j_r\delta^k_t -
%     \delta^i_r\delta^j_t\delta^k_s
% \end{equation*}

% 考虑左侧为$\pm 1$的情况:
% 当$i,j,k$与$r,s,t$分别是$1,2,3$时,容易发现右侧是单位阵的行列式,等于左侧。
% 对$i,j,k$或者$r,s,t$循环移位时,由于相当于行列式交换两次行或者列,左右都不变。
% 对$i,j,k$或者$r,s,t$任意交换时,由于行列式交换一次行或者列,左右都变号。
% 因此所有非零的情况左右相等

% 考虑左侧为$0$:当$i,j,k$或者$r,s,t$一定有任意的重复,可知有两列或者两行重复,左右都是0。

% 总之,上述恒等式是成立的。
% 对上述恒等式$k,t$指标缩并,得到:
% \begin{equation*}
%     \begin{aligned}
%         e^{ijk}e_{rsk}= &
%         \delta^i_r\delta^j_s\delta^k_k +
%         \delta^i_s\delta^j_k\delta^k_r +
%         \delta^i_k\delta^j_r\delta^k_s -
%         \delta^i_k\delta^j_s\delta^k_r -
%         \delta^i_s\delta^j_r\delta^k_k -
%         \delta^i_r\delta^j_k\delta^k_s           \\
%         =               &
%         3\delta^i_r\delta^j_s +
%         \delta^i_s\delta^j_r +
%         \delta^j_r\delta^i_s -
%         \delta^j_s\delta^i_r -
%         3\delta^i_s\delta^j_r -
%         \delta^i_r\delta^j_s
%         \\
%         =               & \delta^i_r\delta^j_s -
%         \delta^i_s\delta^j_r
%     \end{aligned}
% \end{equation*}

% 进一步缩并可得:
% \begin{equation*}
%     \begin{aligned}
%         e^{ijk}e_{rjk}
%         = & \delta^i_r\delta^j_j -
%         \delta^i_j\delta^j_r       \\
%         = & 2\delta^i_r
%     \end{aligned}
% \end{equation*}

% 进一步缩并可得:
% \begin{equation*}
%     \begin{aligned}
%         e^{ijk}e_{ijk}
%         = & 2\delta^i_i
%         =6
%     \end{aligned}
% \end{equation*}

% \subsection*{1.2}
% 在直角坐标展开,根据第一题,有:
% \begin{equation*}
%     \begin{aligned}
%         \bm{u}\times(\bm{v}\times\bm{w})
%         = &
%         u^i\bm{e}_i\times(v^r\bm{e}_r\times w^s\bm{e}_s) \\
%         = &
%         u^i\bm{e}_i\times(v^r w^s e_{rsj}\bm{e}^j)       \\
%         = &
%         u^i v^r w^s e_{rsj}e_{ijk}\bm{e}^k               \\
%         = &
%         u^i v^r w^s (\delta^k_r\delta^i_s-\delta^k_s\delta^i_r)
%         \bm{e}^k                                         \\
%         = &
%         u^i v^k w^i \bm{e}^k- u^i v^i w^k\bm{e}^k        \\
%         = &
%         (\bm{u}\cdot\bm{w})\bm{v}-(\bm{u}\cdot\bm{v})\bm{w}
%     \end{aligned}
% \end{equation*}

% 同样:
% \begin{equation*}
%     \begin{aligned}
%         (\bm{u}\times\bm{v})\cdot(\bm{w}\times\bm{a})
%         = &
%         u^iv^je_{ijk}w^ra^se_{rsk} \\
%         = &
%         u^iv^jw^ra^s(\delta^i_r\delta^j_s -
%         \delta^i_s\delta^j_r)      \\
%         = &
%         u^iv^jw^ia^j-u^iv^jw^ja^i  \\
%         = &
%         (\bm{u}\cdot\bm{w})(\bm{v}\cdot\bm{a})-
%         (\bm{u}\cdot\bm{a})(\bm{v}\cdot\bm{w})
%     \end{aligned}
% \end{equation*}

% \subsection*{1.3}
% 证明:

% 已知单位正交基下混合积定义的轮换性质
% $$
%     [\bm{u},\bm{v},\bm{w}]=
%     (\bm{u}\times\bm{v})\cdot\bm{w}
%     =(\bm{v}\times\bm{w})\cdot\bm{u}
%     =(\bm{w}\times\bm{u})\cdot\bm{v}
% $$

% 因此以下讨论只取轮换等价中的一个情况。

% ~\\

% 先讨论充分性

% 如果$\bm{u},\bm{v},\bm{w}$共面,则不妨设

% $$
%     \bm{w}=a_1\bm{u}+a_2\bm{v}
% $$

% (上式及其轮换形式中必有一个成立,仅讨论此种)

% 则

% $$
%     [\bm{u},\bm{v},\bm{w}]=
%     (\bm{u}\times\bm{v})\cdot\bm{w}
%     =(\bm{u}\times\bm{v})\cdot(a_1\bm{u}+a_2\bm{v})
%     =a_1(\bm{u}\times\bm{v})\cdot\bm{u}+
%     a_2(\bm{u}\times\bm{v})\cdot\bm{v}
% $$

% 容易知道根据叉积的性质,上式最后两项都是0,则

% $$
%     [\bm{u},\bm{v},\bm{w}]=0
% $$

% 以上是充分性的逆否命题,因此得证。

% ~\\

% 再讨论必要性

% 已知$[\bm{u},\bm{v},\bm{w}]=0$

% 假设$\bm{u},\bm{v},\bm{w}$非共面,可知存在$\bm{w'}$

% $$
%     \bm{w}=a_1\bm{u}+a_2\bm{v}+\bm{w'}
% $$

% 且
% $
%     \bm{w'}\neq 0,
%     \bm{w'}\cdot\bm{u}=\bm{w'}\cdot\bm{u}=0
% $。

% 即,必有$\bm{w}$在$span(\bm{u},\bm{v})$的正交空间
% 中的投影$\bm{w'}$非零。

% 又空间只有3维,$span(\bm{u}\times\bm{v})$
% 就是$span(\bm{u},\bm{v})$的正交空间
% (子空间的正交空间的唯一性可以通过构造正交基
% 证明)

% 因此

% $$
%     \left|[\bm{u},\bm{v},\bm{w}]\right|
%     =\left|\bm{w'}\cdot(\bm{u}\times\bm{v})\right|
%     =|\bm{w'}||\bm{u}\times\bm{v}|
% $$

% 根据已知以上最后相乘两项都非0,
% 则知$[\bm{u},\bm{v},\bm{w}]\neq 0$,矛盾。

% 因此必要条件成立。

% \subsection*{1.4}
% 取分量$\bm{x}=(1,1,1), \bm{y}=(0,0,0)$,
% 则$\bm{y}+\bm{x}=(2,2,2)\neq\bm{x}+\bm{y}=(1,1,1)$,不构成阿贝尔群。

% 可以证明,满足实线性空间定义的向量加法,写作实数分量时,由于其交换律、
% 零元、结合律,一定是分量分别相加。下一问的向量加法就如此计算。

% 取分量$\bm{x}=(1,1,1)$,则$(1+1)\bm{x}=(2,1,2)\neq 1\bm{x} + 1\bm{x}=(2,2,2)$,
% 因此不是线性空间的数乘运算。

% \subsection*{1.5}
% 以下讨论都是在$\mathbb{R}^3$中

% \subsubsection*{i}

% C1:共轭对称

% 此处实空间则退化为对称性,交换定义式右端的$x$和$y$发现与此前恒等,则证明。

% C2:数乘线性

% 右端对$x$或者$y$而言都是一次项,无其他幂次,因此满足数乘线性。

% C3:分配律

% 右端对$x$或者$y$而言都是一次项,每一项满足分配律,线性和后仍然满足。

% C4:正定性

% 通过配方可以将其化为正定二次型
% $$
%     (x_1+x_2)(y_1+y_2)+x_1y_1+x_2y_2+x_3y_3
% $$

% 容易看到当$\bm{x}=\bm{y}$上式都是平方项,因此大于等于0且
% 向量为0的时候才是0。

% 或者采用更加
% 数值的方法:

% $$
%     \bm{I'(x,y)}
%     =
%     [\bm{x}]\trans
%     \begin{bmatrix}
%         2 & 1 & 0 \\
%         1 & 2 & 0 \\
%         0 & 0 & 1 \\
%     \end{bmatrix}
%     [\bm{y}]=[\bm{x}]\trans
%     A
%         [\bm{y}]
% $$

% 只需证明矩阵$A$的正定性,由于其严格对角占优且对角线为正则为正定。

% 或者,其特征值有$1,1,3$都是正数因此是正定的。

% 综上,满足内积的定义。

% \subsubsection*{ii}

% 根据线性代数,所有的内积一定有对应双线性函数

% \begin{equation}
%     \bm{I(x,y)}
%     =
%     [\bm{x}]\trans
%     M
%         [\bm{y}]
%     \label{eq:1}
% \end{equation}

% 且满足$M$是正定对称矩阵,而且根据i,容易证明只要有一个正定对称矩阵$M$依此定义出的
% 函数就是内积。

% 因此,所有三维内积算子的集合与所有三维正定对称矩阵的集合是一一映射的。

% 定义任意内积只需对其分量定义(1)形式的函数以及正定对称矩阵$M$即可。

% 如定义矩阵
% $$
%     M=\begin{bmatrix}
%         1 & 0 & 0 \\
%         0 & 2 & 0 \\
%         0 & 0 & 3 \\
%     \end{bmatrix}
% $$

% 则\eqref{eq:1}式就是一个内积。

% \subsection*{1.6}
% 根据直角坐标,通过系数矩阵求逆得到逆变基向量的分量:
% \begin{equation*}
%     \begin{aligned}
%         \bm{g}^1= & \frac{1}{15}(5\bm{e}_1-5\bm{e}_2+5\bm{e}_3)    \\
%         \bm{g}^2= & \frac{1}{15}(-9\bm{e}_1+21\bm{e}_2-18\bm{e}_3) \\
%         \bm{g}^3= & \frac{1}{15}(\bm{e}_1-4\bm{e}_2+7\bm{e}_3)     \\
%     \end{aligned}
% \end{equation*}

% 逆变度量张量的分量则为:
% \begin{equation*}
%     g^{ij}=\bm{g}^i\cdot\bm{g}^j=\frac{1}{75}\begin{bmatrix}
%         25  & -80 & 20  \\
%         -80 & 282 & -73 \\
%         20  & -73 & 22
%     \end{bmatrix}
% \end{equation*}
% % TODO



% \subsection*{1.7}

% 当$\bm{u},\bm{v},\bm{w}$线性相关,左右都是0,显然成立,以下讨论线性无关情况。

% 记$\bm{u},\bm{v},\bm{w}=\bm{g}_1, \bm{g}_2, \bm{g}_3$
% 则一定有逆变基矢量$\bm{g}^1, \bm{g}^2, \bm{g}^3$,使得$\bm{g}^i\cdot\bm{g}_j=\delta^i_j$。
% 则讨论$\bm{a},\bm{b},\bm{c}$的逆变分量展开情况:
% \begin{equation*}
%     \begin{aligned}
%         [\bm{a},\bm{b},\bm{c}]
%         = & [a_i\bm{g}^i,b_j\bm{g}^j,c_k\bm{g}^k] \\
%     \end{aligned}
% \end{equation*}

% 考虑上式右端展开爱因斯坦求和有27项,其中非零的,只有$i,j,k$不等的情况,因此:
% \begin{equation*}
%     \begin{aligned}
%         [\bm{a},\bm{b},\bm{c}]
%         = & [a_i\bm{g}^i,b_j\bm{g}^j,c_k\bm{g}^k] \\
%         = & (
%         a_1b_2c_3 + a_2b_3c_1 + a_3b_1c_2
%         -a_3b_2c_1 - a_2b_1c_3 - a_1b_3c_2
%         )
%         [\bm{g}^1, \bm{g}^2, \bm{g}^3]            \\
%         = & \left|
%         \begin{matrix}
%             a_1 & a_2 & a_3 \\
%             b_1 & b_2 & b_3 \\
%             c_1 & c_2 & c_3 \\
%         \end{matrix}
%         \right|[\bm{g}^1, \bm{g}^2, \bm{g}^3]
%     \end{aligned}
% \end{equation*}

% 由于已知
% \begin{equation*}
%     \begin{aligned}
%         \bm{g}^1= &
%         \frac{1}{\sqrt{g}}(\bm{g}_2\times\bm{g}_3) \\
%         \bm{g}^2= &
%         \frac{1}{\sqrt{g}}(\bm{g}_3\times\bm{g}_1) \\
%         \bm{g}^3= &
%         \frac{1}{\sqrt{g}}(\bm{g}_1\times\bm{g}_2) \\
%     \end{aligned}
% \end{equation*}

% 因此,根据叉积公式(第二题):
% \begin{equation*}
%     \begin{aligned}
%         [\bm{g}^1, \bm{g}^2, \bm{g}^3]= &
%         \frac{1}{g^{\frac{3}{2}}}
%         \left(
%         (\bm{g}_2\times\bm{g}_3)\times
%         (\bm{g}_3\times\bm{g}_1)
%         \right)\cdot
%         (\bm{g}_1\times\bm{g}_2)                                 \\
%         =                               &
%         \frac{1}{g^{\frac{3}{2}}}
%         \left[
%             \left(
%             (\bm{g}_2\times\bm{g}_3)\cdot\bm{g}_1
%             \right)\bm{g}_3-
%             \left(
%             (\bm{g}_3\times\bm{g}_1)\cdot\bm{g}_3
%             \right)\bm{g}_1
%             \right]\cdot
%         (\bm{g}_1\times\bm{g}_2)                                 \\
%         =                               &
%         \frac{1}{g^{\frac{3}{2}}}
%         [\bm{g}_1,\bm{g}_2,\bm{g}_3]\bm{g}_3\cdot
%         (\bm{g}_1\times\bm{g}_2)                                 \\
%         =                               &
%         \frac{1}{g^{\frac{3}{2}}}
%         [\bm{g}_1,\bm{g}_2,\bm{g}_3][\bm{g}_1,\bm{g}_2,\bm{g}_3] \\
%         =                               &
%         \frac{1}{\sqrt{g}}
%     \end{aligned}
% \end{equation*}

% 因此,关于$[\bm{a},\bm{b},\bm{c}]$的展开可得:
% \begin{equation*}
%     \begin{aligned}
%         [\bm{a},\bm{b},\bm{c}]
%         = & \left|
%         \begin{matrix}
%             a_1 & a_2 & a_3 \\
%             b_1 & b_2 & b_3 \\
%             c_1 & c_2 & c_3 \\
%         \end{matrix}
%         \right|\frac{1}{\sqrt{g}} \\
%         \Rightarrow\
%         [\bm{a},\bm{b},\bm{c}][\bm{g}_1,\bm{g}_2,\bm{g}_3]
%         = & \left|
%         \begin{matrix}
%             a_1 & a_2 & a_3 \\
%             b_1 & b_2 & b_3 \\
%             c_1 & c_2 & c_3 \\
%         \end{matrix}
%         \right|
%     \end{aligned}
% \end{equation*}

% 又根据逆变分量的计算方法:
% \begin{equation*}
%     \begin{aligned}
%         a_i & =\bm{a}\cdot\bm{g}_i \\
%         b_i & =\bm{b}\cdot\bm{g}_i \\
%         c_i & =\bm{c}\cdot\bm{g}_i \\
%     \end{aligned}
% \end{equation*}

% 带入上式即为所求证公式。


\end{document}