%!TEX program = xelatex
\documentclass[UTF8,zihao=5]{ctexart}


\title{{\bfseries 第4次作业}}
\author{周涵宇 2022310984}
\date{}

\usepackage[a4paper]{geometry}
\geometry{left=0.75in,right=0.75in,top=1in,bottom=1in}

\usepackage[
UseMSWordMultipleLineSpacing,
MSWordLineSpacingMultiple=1.5
]{zhlineskip}

\usepackage{fontspec}
\setmainfont{Times New Roman}
% \setmonofont{JetBrains Mono}
\setCJKmainfont{仿宋}[AutoFakeBold=true]
\setCJKsansfont{黑体}[AutoFakeBold=true]

% \usepackage{bm}

\usepackage{amsmath,amsfonts}
\usepackage{array}

\newcommand{\bm}[1]{{\mathbf{#1}}}

\newcommand{\trans}[0]{^\mathrm{T}}
\newcommand{\tran}[1]{#1^\mathrm{T}}
\newcommand{\hermi}[0]{^\mathrm{H}}
\newcommand{\conj}[1]{\overline{#1}}

\newcommand*{\av}[1]{\left\langle{#1}\right\rangle}

\newcommand*{\avld}[1]{\frac{\overline{D}#1}{Dt}}

\newcommand*{\pd}[2]{\frac{\partial #1}{\partial #2}}

\newcommand*{\pdcd}[3]
{\frac{\partial^2 #1}{\partial #2 \partial #3}}


\begin{document}

\maketitle

\subsection*{4.1}

考虑:$\bm{g}_i\bm{g}^j=\delta_i^j$,
设出逆变基在标准正交坐标系下的系数,解线性方程即有:

$$
    \begin{aligned}
        \bm{g}^1= & \frac{ 57\bm{e}_1 -17\bm{e}_2 +27\bm{e}_3 -23\bm{e}_4}{261} \\
        \bm{g}^2= & \frac{-45\bm{e}_1+105\bm{e}_2 -90\bm{e}_3 +96\bm{e}_4}{261} \\
        \bm{g}^3= & \frac{-21\bm{e}_1 +20\bm{e}_2 +45\bm{e}_3 -19\bm{e}_4}{261} \\
        \bm{g}^4= & \frac{  9\bm{e}_1 -21\bm{e}_2 +18\bm{e}_3 +33\bm{e}_4}{261} \\
    \end{aligned}
$$

\subsection*{4.2}

以下以$\bm{r}$代表位置矢量。

\subsubsection*{4.2.1}

柱坐标系:$\bm{r}= \cos{(x^2)} x^1 \bm{e}_1 + \sin{(x^2)} x^1 \bm{e}_2 + x^3 \bm{e}_3$

则$\bm{g}_i=\pd{\bm{r}}{x^i}$
$$
    \begin{aligned}
        \bm{g}_1 = & \cos{(x^2)}\bm{e}_1 + \sin{(x^2)}\bm{e}_2        \\
        \bm{g}_2 = & -\sin{(x^2)}x^1\bm{e}_1 + \cos{(x^2)}x^1\bm{e}_2 \\
        \bm{g}_3 = & \bm{e}_3                                         \\
    \end{aligned}
$$

根据$\bm{g}_i\bm{g}^j=\delta_i^j$,即有:
$$
    \begin{aligned}
        \bm{g}^1 = & \cos{(x^2)}\bm{e}_1 + \sin{(x^2)}\bm{e}_2                          \\
        \bm{g}^2 = & -\frac{\sin{(x^2)}}{x^1}\bm{e}_1 + \frac{\cos{(x^2)}}{x^1}\bm{e}_2 \\
        \bm{g}^3 = & \bm{e}_3                                                           \\
    \end{aligned}
$$

因此度量张量:
$$
    \begin{aligned}
        \bm{G}=\bm{g}^1\bm{g}^1 + x^1x^1 \bm{g}^2\bm{g}^2 + \bm{g}^3\bm{g}^3
        =\bm{g}_1\bm{g}_1 + \frac{1}{x^1x^1} \bm{g}_2\bm{g}_2 + \bm{g}_3\bm{g}_3
    \end{aligned}
$$

因此$\det(g_{ij})=\frac{1}{\det(g^{ij})}=x^1x^1=g$

\subsubsection*{4.2.2}

球坐标系:$\bm{r}= x^1\cos{(x^2)}\cos{(x^3)} \bm{e}_1 + x^1\sin{(x^2)}\cos{(x^3)}\bm{e}_2 + x^1\sin{(x^3)}\bm{e}_3$

则$\bm{g}_i=\pd{\bm{r}}{x^i}$
$$
    \begin{aligned}
        \bm{g}^1 = & \cos(x^2)\cos(x^3)\bm{e}_1 + \sin(x^2)\cos(x^3)\bm{e}_2 +\sin(x^3)\bm{e}_3             \\
        \bm{g}^2 = & -x^1\sin(x^2)\cos(x^3)\bm{e}_1 +  x^1\cos(x^2)\cos(x^3)\bm{e}_2                        \\
        \bm{g}^3 = & -x^1\cos(x^2)\sin(x^3)\bm{e}_1 - x^1\sin(x^2)\sin(x^3)\bm{e}_2 +  x^1\cos(x^3)\bm{e}_3 \\
    \end{aligned}
$$

根据$\bm{g}_i\bm{g}^j=\delta_i^j$,即有:
$$
    \begin{aligned}
        \bm{g}^1 = & \cos(x^2)\cos(x^3)\bm{e}_1 + \sin(x^2)\cos(x^3)\bm{e}_2 +\sin(x^3)\bm{e}_3                                     \\
        \bm{g}^2 = & -\frac{\sin(x^2)}{x^1\cos(x^3)}\bm{e}_1 + \frac{\cos(x^2)}{x^1\cos(x^3)}\bm{e}_2                               \\
        \bm{g}^3 = & -\frac{\cos(x^2)\sin(x^3)}{x^1}\bm{e}_1  -\frac{\sin(x^2)\sin(x^3)}{x^1}\bm{e}_2+\frac{\cos(x^3)}{x^1}\bm{e}_3 \\
    \end{aligned}
$$

因此度量张量:
$$
    \begin{aligned}
        \bm{G}= & \bm{g}^1\bm{g}^1 + x^1x^1\cos(x^3)\cos(x^3) \bm{g}^2\bm{g}^2 + x^1x^1\bm{g}^3\bm{g}^3                      \\
        =       & \bm{g}_1\bm{g}_1 + \frac{1}{x^1x^1\cos(x^3)\cos(x^3)} \bm{g}_2\bm{g}_2 + \frac{1}{x^1x^1} \bm{g}_3\bm{g}_3
    \end{aligned}
$$

因此$\det(g_{ij})=\frac{1}{\det(g^{ij})}=\left[x^1x^1\cos(x^3)\right]^2=g$

\subsubsection*{4.2.1}

极坐标系:$\bm{r}= \cos{(x^2)} x^1 \bm{e}_1 + \sin{(x^2)} x^1 \bm{e}_2 $

则$\bm{g}_i=\pd{\bm{r}}{x^i}$
$$
    \begin{aligned}
        \bm{g}_1 = & \cos{(x^2)}\bm{e}_1 + \sin{(x^2)}\bm{e}_2        \\
        \bm{g}_2 = & -\sin{(x^2)}x^1\bm{e}_1 + \cos{(x^2)}x^1\bm{e}_2 \\
    \end{aligned}
$$

根据$\bm{g}_i\bm{g}^j=\delta_i^j$,即有:
$$
    \begin{aligned}
        \bm{g}^1 = & \cos{(x^2)}\bm{e}_1 + \sin{(x^2)}\bm{e}_2                          \\
        \bm{g}^2 = & -\frac{\sin{(x^2)}}{x^1}\bm{e}_1 + \frac{\cos{(x^2)}}{x^1}\bm{e}_2 \\
    \end{aligned}
$$

因此度量张量:
$$
    \begin{aligned}
        \bm{G}=\bm{g}^1\bm{g}^1 + x^1x^1 \bm{g}^2\bm{g}^2
        =\bm{g}_1\bm{g}_1 + \frac{1}{x^1x^1} \bm{g}_2\bm{g}_2
    \end{aligned}
$$

因此$\det(g_{ij})=\frac{1}{\det(g^{ij})}=x^1x^1=g$

\subsection*{4.3}

% [g] is col vector cart for g, so [g_i][beta^i_j'] = [g_j']

4.2已经给出$\bm{g}_i,\,\bm{g}_{i'}$,则根据$\bm{g}_i\beta_{j'}^{i}=\bm{g}_{j'}$,
即有:$\beta_{j'}^{i}=\bm{g}^i\cdot\bm{g}_{j'}$,因此:

$$
    \beta^i_{j'}=
    \left[\begin{array}{ccc} \cos\left(-x^{2'}+x^{2}\right)\,\cos\left(x^{3'}\right) & x^{1'}\,\sin\left(-x^{2'}+x_{2}\right)\,\cos\left(x^{3'}\right) & -x^{1'}\,\cos\left(-x^{2'}+x_{2}\right)\,\sin\left(x^{3'}\right)\\ -\frac{\sin\left(-x^{2'}+x_{2}\right)\,\cos\left(x^{3'}\right)}{x_{1}} & \frac{x^{1'}\,\cos\left(-x^{2'}+x_{2}\right)\,\cos\left(x^{3'}\right)}{x_{1}} & \frac{x^{1'}\,\sin\left(-x^{2'}+x_{2}\right)\,\sin\left(x^{3'}\right)}{x_{1}}\\ \sin\left(x^{3'}\right) & 0 & x^{1'}\,\cos\left(x^{3'}\right) \end{array}\right]_{ij'}
$$

不难发现,由于$x^{2'} = x^2$,
$$
    \beta^i_{j'}=
    \left[\begin{array}{ccc} \cos\left(x^{3'}\right) & 0 & -x^{1'}\,\sin\left(x^{3'}\right)\\ 0 & \frac{x^{1'}\,\cos\left(x^{3'}\right)}{x^{1}} & 0\\ \sin\left(x^{3'}\right) & 0 & x^{1'}\,\cos\left(x^{3'}\right) \end{array}\right]_{ij'}
$$

又由于$x^{1}=x^{1'}\cos(x^{3'})$
$$
    \beta^i_{j'}=
    \left[\begin{array}{ccc} \cos\left(x^{3'}\right) & 0 & -x^{1'}\,\sin\left(x^{3'}\right)\\ 0 & 1 & 0\\ \sin\left(x^{3'}\right) & 0 & x^{1'}\,\cos\left(x^{3'}\right) \end{array}\right]_{ij'}
$$

考虑到$\beta^i_{k'}\beta^{k'}_j=\delta^i_j$,即有:
$$
    \beta^{i'}_j=
    \left[\begin{array}{ccc} \cos\left(x^{3'}\right) & 0 & \sin\left(x^{3'}\right)\\ 0 & 1 & 0\\ -\frac{\sin\left(x^{3'}\right)}{x^{1'}} & 0 & \frac{\cos\left(x^{3'}\right)}{x^{1'}} \end{array}\right]_{i'j}
$$

如果表示表示为$x^i$,有:$x^{1'}=\sqrt{(x^1)^2+(x^3)^2}, x^{2'}=x^2, x^{3'}=\arctan(\frac{x^3}{x^1})$代入即可。

\subsection*{4.4}

首先根据分量协变导数满足:
$\pd{\bm{u}}{x^i}=u_{j;i}\bm{g}^j=u^j_{;i}\bm{g}_j$,其两端张量积$\bm{g}_i$即可发现
最左端是一个二阶张量,右端是按照基矢量的张量积展开的结果,因此协变分量都是张量分量。
又有关系:

$$
    u^i_{;j} = u^i_{,j} + u^k\Gamma_{jk}^i
$$

由于克氏符号不是张量的分量,通过左右两端的坐标变换关系
容易发现,普通偏导数也不是张量的分量,存在不满足坐标变换
关系的例子。

(由于克氏符号在直线坐标下是0,在非直线坐标下是非0,其不是张量的分量,
因此也不满足坐标变换关系)

\subsection*{4.5}

\subsubsection*{4.5.1}
柱坐标下,$[g_{ij}]=diag(1, (x^1)^2, 1)$,是对角阵,考虑
$$
    \Gamma_{ij,k}=
    \frac{1}{2}
    (g_{ik,j} + g_{jk,i} - g_{ij,k})
$$

则$i=k=2,j=1$或者$j=k=2,i=1$时,$\Gamma_{ij,k}=x^1$,
$i=j=2,k=1$时,$\Gamma_{ij,k}=-x^1$,其余分量都是0。


\subsubsection*{4.5.2}
球坐标下,$[g_{ij}]=diag(1, (x^1)^2\cos^2(x^3), (x^1)^2)$,是对角阵,考虑
$$
    \Gamma_{ij,k}=
    \frac{1}{2}
    (g_{ik,j} + g_{jk,i} - g_{ij,k})
$$

% $$
% \Gamma^l_{ij}=g^{kl}
%     \frac{1}{2}
%     (g_{ik,j} + g_{jk,i} - g_{ij,k})
% $$

则$i=k=2,j=1$或者$j=k=2,i=1$时,
$\Gamma_{ij,k}=x^1\cos^2(x^3)$;
当$i=j=2,k=1$时,
$\Gamma_{ij,k}=-x^1\cos^2(x^3)$;
当$i=k=2,j=3$或者$j=k=2,i=3$时,
$\Gamma_{ij,k}=-x^1x^1\cos(x^3)\sin(x^3)$;
当$i=j=2,k=3$时,
$\Gamma_{ij,k}=x^1x^1\cos(x^3)\sin(x^3)$;
当$i=k=3,j=1$或者$j=k=3,i=1$时,
$\Gamma_{ij,k}=x^1$;
当$i=j=3,k=1$时,
$\Gamma_{ij,k}=-x^1$;
其余分量都是0。



\subsubsection*{4.6}

首先由于$\bm{g_i}=\pd{\bm{r}}{x^i}=\pd{\bm{r}}{x^{i'}}\pd{x^i}{x^{i'}}=\bm{g_{i'}}\pd{x^i}{x^{i'}}$
可知$\pd{x^i}{x^{i'}}$就是$\beta^i_{i'}$,反之亦然。


% ! before coord conversion, separate contraction indices with delta

% ! $g_{ij,k} -> g_{i'j,k}$, cannot put $\beta^{i'}_i$ through the partial difference of k!

$$
    \begin{aligned}
        \Gamma^{l'}_{i'j'}
        = & \pd{\bm{g}_{i'}}{x^{j'}}\cdot \bm{g}^{l'}
        =
        \pd{\left(\bm{g}_{i}\pd{x^i}{x^{i'}}\right)}{x^{j'}}
        \cdot\bm{g}^{l}\pd{x^{l'}}{x^{l}}
        =
        \left(
        \pd{\bm{g}_i}{x^{j'}}\pd{x^i}{x^{i'}}
        +
        \bm{g}_i\pdcd{x^i}{x^{i'}}{x^{j'}}
        \right)
        \cdot\bm{g}^{l}\pd{x^{l'}}{x^{l}}             \\
        = &
        \pd{\bm{g}_i}{x^{j'}}\pd{x^i}{x^{i'}}\cdot\bm{g}^{l}\pd{x^{l'}}{x^{l}}
        +\pdcd{x^l}{x^{i'}}{x^{j'}}\pd{x^{l'}}{x^{l}}
        =
        \pd{\bm{g}_i}{x^{j}}\cdot\bm{g}^{l}\beta^j_{j'}\beta^i_{i'}\beta^{l'}_l
        +\pdcd{x^l}{x^{i'}}{x^{j'}}\beta^{l'}_l
        =
        \Gamma^l_{ij}\beta^j_{j'}\beta^i_{i'}\beta^{l'}_l
        +\pdcd{x^l}{x^{i'}}{x^{j'}}\beta^{l'}_l
    \end{aligned}
$$

其中老坐标对新坐标的二阶导数不一定都是0,而且对于一般的函数,
坐标转换系数作为一阶导数与二阶导数在给定的一点处可无关,因此总
存在这一项非零的构造,因此克氏符号不满足坐标变换关系,不是一个张量。

\subsection*{4.7}

对于任意维数,可知:
$$
    \sqrt{g}=\sqrt{\det(g_{ij})}=\pm[\bm{g}_1,\bm{g}_2,...\bm{g}_n]
$$

其中中括号运算是三维混合积的推广,此处称作广义混合积,使得:
$$
    [\bm{g}_1,\bm{g}_2,...,\bm{g}_n]
    =\bm{g}_1\cdot \bm{e}^{i_1}
    \bm{g}_2\cdot \bm{e}^{i_2}...
    \bm{g}_n\cdot \bm{e}^{i_n}
    \delta_{1,2,...n}^{i_1,i_2,...i_n}
$$

其中,
$$
    \delta_{1,2,...n}^{i_1,i_2,...i_n}
    =
    \left|\begin{matrix}
        \delta_1^{i_1} & \delta_1^{i_2} & \cdots & \delta_1^{i_n} \\
        \delta_2^{i_1} & \delta_2^{i_2} & \cdots & \delta_2^{i_n} \\
        \vdots         & \vdots         & \ddots & \vdots         \\
        \delta_n^{i_1} & \delta_n^{i_2} & \cdots & \delta_n^{i_n} \\
    \end{matrix}\right|
$$

这个结论可以用矩阵理论得到,
同时根据行列式的性质可知,当其中两个向量相同时
广义混合积运算为0。同时广义混合基对任意输入向量线性(进行坐标变换,因而具有分配律。

方便讨论,设$[\bm{g}_1,\bm{g}_2,...\bm{g}_n]>0$,也就是说取一种“手性”。
另外一种手性的推导只相差一个符号。

观察$\sqrt{g}$的展开,其中标正基和坐标无关,且考虑线性性质、分配律,
可知,对其求导后的代数形式与n个变量的乘积的导数相同:

$$
    \begin{aligned}
        \pd{\sqrt{g}}{x^i}= &
        [\pd{\bm{g}_1}{x^i}, \bm{g}_2,...\bm{g}_n]
        +[\bm{g}_1, \pd{\bm{g}_2}{x^i},...\bm{g}_n]+...
        +[\bm{g}_1, \bm{g}_2,...\pd{\bm{g}_n}{x^i}]    \\
        =                   &
        [\Gamma^l_{1i}\bm{g}_l, \bm{g}_2,...\bm{g}_n]
        +[\bm{g}_1, \Gamma^l_{2i}\bm{g}_l,...\bm{g}_n]+...
        +[\bm{g}_1, \bm{g}_2,...\Gamma^l_{ni}\bm{g}_l] \\
        =                   &
        [\Gamma^1_{1i}\bm{g}_1, \bm{g}_2,...\bm{g}_n]
        +[\bm{g}_1, \Gamma^2_{2i}\bm{g}_2,...\bm{g}_n]+...
        +[\bm{g}_1, \bm{g}_2,...\Gamma^n_{ni}\bm{g}_n]\\
        =                   &
        \Gamma^1_{1i}[\bm{g}_1, \bm{g}_2,...\bm{g}_n]
        +\Gamma^2_{2i}[\bm{g}_1, \bm{g}_2,...\bm{g}_n]+...
        +\Gamma^n_{ni}[\bm{g}_1, \bm{g}_2,...\bm{g}_n]\\
        =                   &
        \Gamma^l_{li}[\bm{g}_1, \bm{g}_2,...\bm{g}_n]
    \end{aligned}
$$

其中第三个等号是因为展开缩并项后,带有重复向量的广义混合积运算都是0,仅剩一项。
因此:
$$
\pd{\sqrt{g}}{x^i}=\Gamma^l_{li}\sqrt{g}
\Rightarrow
\Gamma^l_{li}=\frac{1}{\sqrt{g}}\pd{\sqrt{g}}{x^i}
=\pd{\ln\sqrt{g}}{x^i}
$$



% \subsection*{3.1}

% 方便起见按照分量形式:

% $$
% A_{ij}=-\epsilon_{ijk}a^k
% $$

% 由于$\bm{A}$是反对称的,其特征值只有$\pm i\alpha$以及$0$。
% 考虑$A_{ij}a^j=-\epsilon_{ijk}a^ka^j=0$,可知,$\bm{a}$是$\bm{a}$的一个特征向量,对应特征值为0。
% 由于另外两个特征值与0不同,其特征向量与这个特征向量线性无关,因此0的特征向量都是$\bm{a}$共线的。

% \subsection*{3.2}

% 考虑由于$\bm{S}$是一个线性的映射,因此对$\bm{e}_1$取反,则$\bm{r}_1$取反,因此$\gamma$取反。
% 所以,只讨论其极大值即可。因此,只考虑$\bm{r}_1,\bm{r}_2$的夹角极值即可。

% 不妨求$\bm{S}$的主轴和特征值$s_1\geq s_2 \geq s_3$,沿主轴建立单位直角坐标$\bm{g}_i=\bm{g}^i$,
% 记$\bm{e}_1=a_i\bm{g}_i, \bm{e}_2=b_i\bm{g}_i$,代入即有:

% \begin{equation}
%     \begin{aligned}
%         \cos{\left\langle \bm{r}_1,\bm{r}_2 \right\rangle}
%         =\frac{s_1^2a_1b_1 + s_2^2a_2b_2 + s_3^2a_3b_3}{
%             \sqrt{s_1^2a_1^2 + s_2^2a_2^2+s_3^2a_3^2}
%             \sqrt{s_1^2b_1^2 + s_2^2b_2^2+s_3^2b_3^2}
%         }
%     \end{aligned}
% \end{equation}

% 记$r_1=\sqrt{s_1^2a_1^2 + s_2^2a_2^2+s_3^2a_3^2}, r_2=\sqrt{s_1^2b_1^2 + s_2^2b_2^2+s_3^2b_3^2}$

% 则:

% \begin{equation}
%     \left\{
%         \begin{aligned}
%             \pd{\cos{\left\langle \bm{r}_1,\bm{r}_2 \right\rangle}}{a_1}
%             = & \frac{b_1s_1^2}{r_1r_2} -
%             \frac{a_1s_1^2(s_1^2a_1b_1 + s_2^2a_2b_2 + s_3^2a_3b_3)}{r_1^3r_2}\\
%             \pd{\cos{\left\langle \bm{r}_1,\bm{r}_2 \right\rangle}}{a_2}
%             = & \frac{b_2s_2^2}{r_1r_2} -
%             \frac{a_2s_2^2(s_1^2a_1b_1 + s_2^2a_2b_2 + s_3^2a_3b_3)}{r_1^3r_2}\\
%             \pd{\cos{\left\langle \bm{r}_1,\bm{r}_2 \right\rangle}}{a_3}
%             = & \frac{b_3s_3^2}{r_1r_2} -
%             \frac{a_3s_3^2(s_1^2a_1b_1 + s_2^2a_2b_2 + s_3^2a_3b_3)}{r_1^3r_2}\\
%             \pd{\cos{\left\langle \bm{r}_1,\bm{r}_2 \right\rangle}}{b_1}
%             = & \frac{a_1s_1^2}{r_1r_2} -
%             \frac{b_1s_1^2(s_1^2a_1b_1 + s_2^2a_2b_2 + s_3^2a_3b_3)}{r_1r_2^3}\\
%             \pd{\cos{\left\langle \bm{r}_1,\bm{r}_2 \right\rangle}}{b_2}
%             = & \frac{a_2s_2^2}{r_1r_2} -
%             \frac{b_2s_2^2(s_1^2a_1b_1 + s_2^2a_2b_2 + s_3^2a_3b_3)}{r_1r_2^3}\\
%             \pd{\cos{\left\langle \bm{r}_1,\bm{r}_2 \right\rangle}}{b_3}
%             = & \frac{a_3s_3^2}{r_1r_2} -
%             \frac{b_3s_3^2(s_1^2a_1b_1 + s_2^2a_2b_2 + s_3^2a_3b_3)}{r_1r_2^3}\\
%         \end{aligned}
%     \right.
% \end{equation}

% 设拉格朗日乘子$L_1,L_2,K$,对应问题的全部约束(垂直且都是单位长度):

% \begin{equation}
%     \left\{
%         \begin{aligned}
%             F_1= & a_1^2+a_2^2+a_3^2-1\\
%             F_2= & b_1^2+b_2^2+b_3^2-1\\
%             G= & a_1b_1 + a_2b_2 + a_3b_3\\
%         \end{aligned}
%     \right.
% \end{equation}

% 则目标函数为$M = \cos{\left\langle \bm{r}_1,\bm{r}_2 \right\rangle} - L_1F_1 - L_2F_2 - KG$,
% 求解$\pd{M}{a_i}=0, \pd{M}{b_i}=0, F_1=0, F_2=0, G=0$,则可以得到夹角的极值点可能的位置。

% 求解:
% \begin{equation}
%     \left\{
%         \begin{aligned}
%             \frac{b_1s_1^2}{r_1r_2} -
%             \frac{a_1s_1^2(s_1^2a_1b_1 + s_2^2a_2b_2 + s_3^2a_3b_3)}{r_1^3r_2} - 2L_1a_1 - Kb_1 = & 0\\
%             \frac{b_2s_2^2}{r_1r_2} -
%             \frac{a_2s_2^2(s_1^2a_1b_1 + s_2^2a_2b_2 + s_3^2a_3b_3)}{r_1^3r_2} - 2L_1a_2 - Kb_2 = & 0\\
%             \frac{b_3s_3^2}{r_1r_2} -
%             \frac{a_3s_3^2(s_1^2a_1b_1 + s_2^2a_2b_2 + s_3^2a_3b_3)}{r_1^3r_2} - 2L_1a_3 - Kb_3 = & 0\\
%             \frac{a_1s_1^2}{r_1r_2} -
%             \frac{b_1s_1^2(s_1^2a_1b_1 + s_2^2a_2b_2 + s_3^2a_3b_3)}{r_1r_2^3} - 2L_2b_1 - Ka_1 = & 0\\
%             \frac{a_2s_2^2}{r_1r_2} -
%             \frac{b_2s_2^2(s_1^2a_1b_1 + s_2^2a_2b_2 + s_3^2a_3b_3)}{r_1r_2^3} - 2L_2b_2 - Ka_2 = & 0\\
%             \frac{a_3s_3^2}{r_1r_2} -
%             \frac{b_3s_3^2(s_1^2a_1b_1 + s_2^2a_2b_2 + s_3^2a_3b_3)}{r_1r_2^3} - 2L_2b_3 - Ka_3 = & 0\\
%             a_1^2+a_2^2+a_3^2-1 = & 0 \\
%             b_1^2+b_2^2+b_3^2-1 = & 0\\
%             a_1b_1 + a_2b_2 + a_3b_3 = & 0\\
%         \end{aligned}
%     \right.
% \end{equation}

% 方程关于坐标的下标是对称的,因此轮换$a_i,b_i,s_i$的下标同样是解。

% 方程的解只有:
% $$
% \begin{aligned}
%     (a_1,a_2,a_3)=\pm(\frac{1}{\sqrt{2}},\frac{1}{\sqrt{2}}, 0),\ (b_1,b_2,b_3)=\pm(-\frac{1}{\sqrt{2}},\frac{1}{\sqrt{2}}, 0) & \\
%     (a_1,a_2,a_3)=\pm(\frac{1}{\sqrt{2}},\frac{1}{\sqrt{2}}, 0),\ (b_1,b_2,b_3)=\mp(-\frac{1}{\sqrt{2}},\frac{1}{\sqrt{2}}, 0) & \\
% \end{aligned}
% $$
% 以及$a,b$互换和$1,2,3$轮换的结果,共24种。

% 给出的$\cos{\left\langle \bm{r}_1,\bm{r}_2 \right\rangle}$分别是:
% $$
% \cos{\left\langle \bm{r}_1,\bm{r}_2 \right\rangle} = \pm\frac{s_1^2-s_2^2}{s_1^2+s_2^2}
% $$
% 及其轮换。
% 综上,极值或者鞍点只在$\bm{e}_1,\bm{e}_2$都在某两个主轴的平面上且与主轴夹角$45^\circ$出现,
% 最大值、最小值分别是$\frac{\pi}{2} - \arccos{(\pm\frac{s_1^2-s_3^2}{s_1^2+s_3^2})}$,此时$\bm{e}_1, \bm{e}_2$在$s_1,s_3$的主轴平面内,
% 与主轴成$45^\circ$。


% \subsection*{3.3}

% 不妨在标准正交基给出:$\bm{A}=\bm{e}_i\bm{e}_jA_{ij}$,
% 则$\bm{A}\trans=\bm{e}_j\bm{e}_iA_{ij}=-\bm{e}_i\bm{e}_jA_{ij}$,因此$A_{ij}=-A_{ji}$,因此
% $A_{1,1}=A{2,2}=0, A_{1,2}=-A_{2,1}$,可记$\bm{A}=m(\bm{e}_2\bm{e}_1-\bm{e}_1\bm{e}_2)$,
% 因此$det(\bm{A})=m^2, det(\bm{A}-\lambda\bm{I})=\lambda^2+m^2$,因此有共轭特征值$\pm mi :=\pm ai$,
% 因此题述都成立。

% \subsection*{3.4}
% 此处引用郑老师给出的各向同性张量函数在三维情况下的表现定理,
% 由于$\theta \bm{L}$有特征值$\pm i\theta, 0$则:

% \begin{equation}
%     \begin{aligned}
%         e^{\theta \bm{L}} = &
%         \left(-\frac{e^{i\theta}}{2}
%         -\frac{e^{i\theta}}{2} + 1
%         \right) \bm{L}^2 
%         + \left(
%             i\frac{e^{i\theta}}{2}
%             -i\frac{e^{-i\theta}}{2}
%         \right) \bm{L}  + \bm{I}\\
%         = & (1 - \cos \theta)\bm{L}^2 - \sin\theta\bm{L}
%     \end{aligned}
% \end{equation}

% 那么对于任意的$\bm{x}$,
% $e^{\theta \bm{L}}\cdot\bm{x} = (1 - \cos \theta)\bm{r}\times(\bm{r}\times \bm{x}) - \sin\theta \bm{r}\times \bm{x} + \bm{x}$。

% 记$\bm{r}$的正交的平面是$P$,那么$\bm{r}\times(\bm{r}\times \bm{x})$与$\sin\theta \bm{r}$都在这个平面内,
% 且其长度都是$\bm{x}$在$P$上投影的长度,且其相互垂直,
% 构成$P$上的一组正交基。几何分析可知,
% $(1 - \cos \theta)\bm{r}\times(\bm{r}\times \bm{x}) - \sin\theta \bm{r}\times \bm{x}$
% 就是$\bm{x}$与$R_\theta(\bm{x})$的差,旋转方向满足右手定则,因此得证。

% 更为符号地,记$\bm{y} = e^{\theta \bm{L}}\cdot\bm{x}$,则

% $$
% \begin{aligned}
%     \bm{y}\cdot\bm{y} = &(1 - \cos \theta)^2
% ((\bm{r}\cdot\bm{x})^2+\bm{x}\cdot\bm{x}-2(\bm{r}\cdot\bm{x})^2) +
% (\sin\theta)^2 (\bm{x}\cdot\bm{x}-(\bm{r}\cdot\bm{x})^2)+\bm{x}\cdot\bm{x}
% + 2(1 - \cos \theta)[(\bm{r}\cdot\bm{x})^2-\bm{x}\cdot\bm{x}]\\
% =&\bm{x}\cdot\bm{x}
% \end{aligned}
% $$

% 因此长度不变;

% $$
% \begin{aligned}
%     \bm{y}\cdot\bm{y} = &(1 - \cos \theta)^2
% ((\bm{r}\cdot\bm{x})^2+\bm{x}\cdot\bm{x}-2(\bm{r}\cdot\bm{x})^2) +
% (\sin\theta)^2 (\bm{x}\cdot\bm{x}-(\bm{r}\cdot\bm{x})^2)+\bm{x}\cdot\bm{x}
% + 2(1 - \cos \theta)[(\bm{r}\cdot\bm{x})^2-\bm{x}\cdot\bm{x}]\\
% =&\bm{x}\cdot\bm{x}
% \end{aligned}
% $$

% 因此长度不变;

% 类似可以验证
% $$
% \bm{y}\cdot\bm{x}=\cos{\theta}\bm{x}\cdot\bm{x}
% $$
% 因此夹角正确;

% $$
% \bm{x}\times\bm{y}=\sin{\theta}(\bm{x}\cdot\bm{x}) \bm{r}
% $$
% 因此旋转方向正确。

% \subsubsection*{3.5}

% 由于反对称性:

% $$
% e^{\bm{A}\trans}
% =\sum_{n=0}^\infty{\frac{(-1)^n\bm{A}^n}{n!}}
% $$

% 考察函数$e^{-x}$在0的幂级数展开,与之正好形式相同,
% 因此可以给出$e^{{\bm{A}\trans}}=e^{-\bm{A}}$。

% 下面证明,对于任意二阶张量$\bm{B}$,有$e^{\bm{B}}\cdot e^{-\bm{B}}=\bm{I}$。

% $$
% e^{\bm{B}}\cdot e^{-\bm{B}}=\sum_{n=0}^\infty\sum_{m=0}^\infty{
%     \frac{(-1)^n\bm{B}^{m+n}}{n!m!}
% }
% =
% \sum_{k=0}^\infty\sum_{n=0}^k{
%     \frac{(-1)^n\bm{B}^{k}}{n!(k-n)!}
% }
% =
% \sum_{k=0}^\infty(1-1)^k\bm{B}^k = \bm{I}
% $$

% 推导使用了$(1-1)^k$的二项式展开。即证得。

% \subsection*{3.6}

% 对称张量一定有$\bm{A}=\bm{V}bm{D}\bm{V}\trans$,
% 其中$\bm{V}$是正交二阶张量,$\bm{D}$是$\sum_i\lambda_i\bm{e}_i\bm{e}_i$(不爱因斯坦求和),
% 同时根据第一不变量,$tr\bm{A}=tr\bm{D}=\sum_i\lambda_i$。因此:
% $$
% \begin{aligned}
%     \det(e^{\bm{A}})= &
% \det\left(
%     \sum_{n=0}^\infty{\frac{\bm{A}^n}{n!}}
% \right)
% =\det(\bm{V})
% \det\left(
%     \sum_{n=0}^\infty{\frac{\bm{D}^n}{n!}}
% \right)\det(\bm{V}\trans)
% =
% \det\left(
%     \sum_{n=0}^\infty{
%         diag(\lambda_i^n/n! ,i)
%     }
% \right)
% =
% \det\left(
%     diag(
%         \sum_{n=0}^\infty\frac{\lambda_i^n}{n!},i
%     )
% \right)\\
% = &
% \det\left(
%     diag(
%         \sum_{n=0}^\infty\frac{\lambda_i^n}{n!},i
%     )
% \right)
% =\det\left(
%     diag(e^\lambda_i,i)
%     \right)
%     =\prod_i e^{\lambda_i}
%     =e^{\sum_i\lambda_i}
%     =e^{tr\bm{A}}
% \end{aligned} 
% $$

% 其中$diag(a_i, i)=\sum_i a_i\bm{e}_i\bm{e}_i$(无爱因斯坦求和)。

% \subsection*{3.7}

% 若已知$\bm{A}\cdot\bm{B}=\bm{B}\cdot\bm{A}$,则点乘构成阿贝尔群,二项式定理对其成立:
% $(\bm{A}+\bm{B})^k=\sum_{n=0}^k\frac{k!\bm{A}^n\bm{B}^{k-n}}{n!(k-n)!}$。
% 因此有:

% $$
% e^{\bm{A}}\cdot e^\bm{B}
% =\left(\sum_{n=0}^\infty\frac{\bm{A}^n}{n!}\right)\cdot
% \left(\sum_{n=0}^\infty\frac{\bm{B}^n}{n!}\right)
% =\sum_{n=0}^\infty\sum_{m=0}^\infty\frac{\bm{A}^n\cdot\bm{B}^m}{n!m!}
% =\sum_{k=0}^\infty\sum_{n=0}^k\frac{\bm{A}^n\cdot\bm{B}^{k-n}}{n!{k-n}!k!}
% =\sum_{k=0}^\infty\frac{(\bm{A}+\bm{B})^k}{k!}
% =e^{\bm{A}+\bm{B}}
% $$

% 因此可交换是充分条件。






\end{document}