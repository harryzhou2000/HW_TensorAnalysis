%!TEX program = xelatex
\documentclass[UTF8,zihao=5]{ctexart}


\title{{\bfseries 第6次作业}}
\author{周涵宇 2022310984}
\date{}

\usepackage[a4paper]{geometry}
\geometry{left=0.75in,right=0.75in,top=1in,bottom=1in}

\usepackage[
UseMSWordMultipleLineSpacing,
MSWordLineSpacingMultiple=1.5
]{zhlineskip}

\usepackage{fontspec}
\setmainfont{Times New Roman}
% \setmonofont{JetBrains Mono}
\setCJKmainfont{仿宋}[AutoFakeBold=true]
\setCJKsansfont{黑体}[AutoFakeBold=true]

% \usepackage{bm}

\usepackage{amsmath,amsfonts}
\usepackage{array}

\newcommand{\bm}[1]{{\mathbf{#1}}}

\newcommand{\trans}[0]{^\mathrm{T}}
\newcommand{\tran}[1]{#1^\mathrm{T}}
\newcommand{\hermi}[0]{^\mathrm{H}}
\newcommand{\conj}[1]{\overline{#1}}

\newcommand*{\av}[1]{\left\langle{#1}\right\rangle}

\newcommand*{\avld}[1]{\frac{\overline{D}#1}{Dt}}

\newcommand*{\pd}[2]{\frac{\partial #1}{\partial #2}}

\newcommand*{\pdcd}[3]
{\frac{\partial^2 #1}{\partial #2 \partial #3}}


\begin{document}

\maketitle

\subsection*{6.1}

坐标$\xi_i$的基函数就是$\bm{e}^i$,柱坐标$x^i$的基函数设为$\bm{g}_i$,
记变形后的直角坐标是$x_i$,则有:
$$
    \bm{g}_j=\bm{e}_i\pd{x_i}{x^j}
$$
给出基函数的具体形式:

$$
    [[\bm{g}_1,\bm{g}_2,\bm{g}_3]]=[[\bm{e}_1,\bm{e}_2,\bm{e}_3]]
    \begin{bmatrix}
        \cos{x^2} & -x^1\sin{x^2} & 0 \\
        \sin{x^2} & x^1\cos{x^2}  & 0 \\
        0         & 0             & 1 \\
    \end{bmatrix}
$$
其中双括号代表按照矩阵运算进行展开,最后一项矩阵就是
$\left[\pd{x_i}{x^j}\right]_{ij}$。

同时给出逆变基:
$$
    [[\bm{g}^1,\bm{g}^2,\bm{g}^3]]=[[\bm{e}_1,\bm{e}_2,\bm{e}_3]]
    \begin{bmatrix}
        \cos{x^2}              & \sin{x^2}             & 0 \\
        -\frac{\sin{x^2}}{x^1} & \frac{\cos{x^2}}{x^1} & 0 \\
        0                      & 0                     & 1 \\
    \end{bmatrix}
$$
最后一项矩阵就是
$\left[\pd{x^i}{x_j}\right]_{ij}$。

考虑变形梯度张量:
$$
    \bm{F}=\pd{\bm{x}}{\bm{\xi}}
    =
    \pd{x^i}{\xi_A}\bm{g}_i\bm{e}_A
$$
大写的上下标用来表示在变形前(Lagrange)的基矢量和分量,但是转换为
变形后场的直角分量时仍然这么写,特殊说明。

其中坐标的偏导数是:
$$
    \pd{x^i}{\xi_A}=\begin{bmatrix}
        1 & 0             & 0 \\
        0 & \frac{1}{r_0} & 0 \\
        0 & 0             & 1 \\
    \end{bmatrix}^{iA}
$$
这个矩阵是混合坐标(直角-柱坐标)形式的变形梯度分量。
如果写作直角坐标分量形式,则有(忽略上下标区别):
$$
    \bm{F}=\pd{x^i}{\xi_A}\bm{e_k}\pd{x_k}{x^i}\bm{e}_A
    =\left[
        \pd{x_k}{x^i}\pd{x^i}{\xi_A}
        \right]_{kA}\bm{e_k}\bm{e}_A
$$
直角坐标分量的矩阵是:
$$
    \left[
        \pd{x_k}{x^i}\pd{x^i}{\xi_A}
        \right]_{kA}=
    \left[
        \pd{x_k}{\xi_A}
        \right]_{kA}=\begin{bmatrix}
        \cos{x^2} & -\frac{x^1\sin{x^2}}{r_0} & 0 \\
        \sin{x^2} & \frac{x^1\cos{x^2}}{r_0}  & 0 \\
        0         & 0                         & 1 \\
    \end{bmatrix}_{kA}
$$

考虑$\bm{F}^{-1}=(F^{-1})_{Aj}\bm{e}^{A}\bm{g}^{j}$:
$$
    \delta^i_j\bm{g}_i\bm{g}^j=\bm{I}=\bm{F}\cdot\bm{F}^{-1}
    =\pd{x^i}{\xi_A}(F^{-1})_{Bj}\bm{g}_i\bm{e}_A\cdot\bm{e}^{B}\bm{g}^{j}
    =\pd{x^i}{\xi_A}(F^{-1})_{Aj}\bm{g}_i\bm{g}^{j}
$$

则根据分量关系解得:
$$
    [(F^{-1})_{Aj}]=\begin{bmatrix}
        1 & 0   & 0 \\
        0 & r_0 & 0 \\
        0 & 0   & 1 \\
    \end{bmatrix}_{Aj}
$$
这是从柱坐标到直角坐标的变形梯度混合分量。求解直角坐标下的分量,直接从直角坐标展开
$\bm{F}^{-1}=(^CF^{-1})_{Aj}\bm{e}^{A}\bm{e}^{j}$出发:
$$
    \delta^i_j\bm{e}_i\bm{e}^j=\bm{I}=\bm{F}\cdot\bm{F}^{-1}
    =\pd{x_i}{\xi_A}(^CF^{-1})_{Bj}\bm{e}_i\bm{e}_A\cdot\bm{e}^{B}\bm{e}^{j}
    =\pd{x^i}{\xi_A}(^CF^{-1})_{Aj}\bm{e}_i\bm{e}^{j}
$$

则可以解得:
$$
    [(^CF^{-1})_{Aj}]=
    \begin{bmatrix}
        \cos{x^2}                 & \sin{x^2}                & 0 \\
        -\frac{r_0\sin{x^2}}{x^1} & \frac{r_0\cos{x^2}}{x^1} & 0 \\
        0                         & 0                        & 1 \\
    \end{bmatrix}
$$
这就是直角坐标下的逆变形梯度张量的分量。
再求转置则有:
$$
    \bm{F}^{-T}=(^CF^{-1})_{Aj}\bm{e}^{j}\bm{e}^{A}
    =(F^{-1})_{Aj}\bm{g}^{j}\bm{e}^{A}
$$

方便起见在直角坐标下求$\bm{U},\bm{V}$,也就是求直角分量满足:
$$
    \bm{U}=U_{AB}\bm{e}_A\bm{e}_B,\ \ \bm{V}=V_{ij}\bm{e}_i\bm{e}_j
$$

根据极分解计算方法,有$\bm{F}^T\bm{F}=\bm{C}=\bm{U}^2$
考虑直角分量
$$
    \bm{C}=C_{AB}\bm{e}_A\bm{e}_B=\pd{x_k}{\xi_A}\pd{x_k}{\xi_B}\bm{e}_A\bm{e}_B
$$
可得
$$
    [C_{AB}]=\begin{bmatrix}
        1 & 0                     & 0 \\
        0 & \frac{x^1x^1}{r_0r_0} & 0 \\
        0 & 0                     & 1 \\
    \end{bmatrix}
$$
直接可见特征值和特征向量,明显有特征向量为$\bm{N}_A=\bm{e}_A$,则有:
$$
    [U_{AB}]=\begin{bmatrix}
        1 & 0               & 0 \\
        0 & \frac{x^1}{r_0} & 0 \\
        0 & 0               & 1 \\
    \end{bmatrix}_{AB}
$$
注意这是直角坐标的分量。

又根据$\bm{F}\bm{F}^{T}=\bm{c}=\bm{V}^2$,
$$
    \bm{c}=c_{ij}\bm{e}_i\bm{e}_j=\pd{x_i}{\xi_A}\pd{x_j}{\xi_A}\bm{e}_i\bm{e}_j
$$
计算得:
$$
    c_{ij}=\begin{bmatrix}
        \cos^2{x^2} + \frac{x^1x^1}{r_0r_0}\sin^2{x^2}         &
        \left(1-\frac{x^1x^1}{r_0r_0}\right)\cos{x^2}\sin{x^2} & 0     \\
        \left(1-\frac{x^1x^1}{r_0r_0}\right)\cos{x^2}\sin{x^2} &
        \sin^2{x^2} + \frac{x^1x^1}{r_0r_0}\cos^2{x^2}         & 0     \\
        0                                                      & 0 & 1 \\
    \end{bmatrix}_{ij}
$$

可知,其中特征值为$[\lambda_1^2,\lambda_2^2,\lambda_3^2]=[1,\frac{x^1x^1}{r_0r_0},1]$,
特征向量为
$$
    \begin{aligned}
        \bm{n}_1 & =-\sin{x^2}\bm{e}_1 + \cos{x^2}\bm{e}_2 \\
        \bm{n}_2 & = \cos{x^2}\bm{e}_1 + \sin{x^2}\bm{e}_2 \\
        \bm{n}_3 & = \bm{e}_3                              \\
    \end{aligned}
$$

则$\bm{V}=\lambda_1\bm{n}_1\bm{n}_1 + \lambda_2\bm{n}_2\bm{n}_2 + \lambda_3\bm{n}_3\bm{n}_3$
可得分量为:
$$
    [V_{ij}]=\begin{bmatrix}
        1 + \frac{x^1-r_0}{r_0}\sin^2{x^2}               &
        -\frac{x^1-r_0}{r_0}\sin{x^2}\cos{x^2}           & 0     \\
        -\frac{x^1-r_0}{r_0}\sin{x^2}\cos{x^2}           &
        \frac{x^1}{r_0} - \frac{x^1-r_0}{r_0}\sin^2{x^2} & 0     \\
        0                                                & 0 & 1
    \end{bmatrix}_{ij}
$$
这是$\bm{V}$的直角分量。

\subsection*{6.2}
根据文献,对于对称的应力和应变,可以将其分别表示成6维向量,由对角部分和非对角部分组成,
存在约束,等价于$\bm{S}$存在大于零维的化零空间,
也就是说是对称矩阵$[s_{ij}]$存大于零维在化零空间,此时由于约束存在,
刚度在这个约束的空间(也就是化零空间)内是无穷大,因此不存在有限刚度。

当然,根据展开的6维矩阵形式,也方便证明柔度矩阵非奇异与存在刚度矩阵是等价的。
因为若存在,对于任意的$\varepsilon_i=s_{ij}\sigma_j$,
都有$\sigma_i=c_{ij}\varepsilon_j$,展开的应力、应变向量是任意向量(因为
各分量独立)。因此,$[s_{ij}]$与$[c_{ij}]$互逆。如果非奇异则其逆就是刚度矩阵,
如果奇异则由于不可逆而没有刚度矩阵。

对本题给出的$s_{ij}$进行高斯消元或者QR分解都容易发现其化零空间的维数是1,也就是奇异,
化零空间(特征值为0的特征空间)由向量
$$
    (\sigma_0)_i=\left[\begin{array}{c} -\frac{63}{38} \\
            -\frac{28}{19}     \\ -\frac{21}{38}\\
            \frac{9}{38}       \\ \frac{105}{38}\\ 1\end{array}\right]_i
$$
张成,也就是说对应的应力(按照论文展开)(及其倍数):
$$
    [(\sigma_0)_{ij}]=
    \frac{1}{38}\begin{bmatrix}
        -63         & 38\sqrt{2} & 105\sqrt{2} \\
        38\sqrt{2}  & -56        & 9\sqrt{2}    \\
        105\sqrt{2} & 9\sqrt{2}  & -21         \\
    \end{bmatrix}
$$
下,应变是0。
因此不存在刚度矩阵。

也就是说,(根据论文)上述本构存在约束$(\sigma_0)_{ij}\varepsilon_{ij}=0$,
这导致了柔度的奇异性。



\end{document}