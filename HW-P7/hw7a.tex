%!TEX program = xelatex
\documentclass[UTF8,zihao=5]{ctexart}


\title{{\bfseries 第7次作业}}
\author{周涵宇 2022310984}
\date{}

\usepackage[a4paper]{geometry}
\geometry{left=0.75in,right=0.75in,top=1in,bottom=1in}

\usepackage[
UseMSWordMultipleLineSpacing,
MSWordLineSpacingMultiple=1.5
]{zhlineskip}

\usepackage{fontspec}
\setmainfont{Times New Roman}
% \setmonofont{JetBrains Mono}
\setCJKmainfont{仿宋}[AutoFakeBold=true]
\setCJKsansfont{黑体}[AutoFakeBold=true]

% \usepackage{bm}

\usepackage{amsmath,amsfonts}
\usepackage{array}

\newcommand{\bm}[1]{{\mathbf{#1}}}

\newcommand{\trans}[0]{^\mathrm{T}}
\newcommand{\tran}[1]{#1^\mathrm{T}}
\newcommand{\hermi}[0]{^\mathrm{H}}
\newcommand{\conj}[1]{\overline{#1}}

\newcommand*{\av}[1]{\left\langle{#1}\right\rangle}

\newcommand*{\avld}[1]{\frac{\overline{D}#1}{Dt}}

\newcommand*{\pd}[2]{\frac{\partial #1}{\partial #2}}

\newcommand*{\pdcd}[3]
{\frac{\partial^2 #1}{\partial #2 \partial #3}}


\begin{document}

\maketitle

\subsection*{7.1}

根据微分形式的定义,有
$$
    \bm{G}'(F)*d\bm{F}=d\bm{G}
$$
其中$d\bm{F}$是趋近于0的任意张量,即$d\bm{F}=h\bm{H},h\rightarrow 0,\forall \bm{H}$,
而$d\bm{G}=\bm{G}(\bm{F}+d\bm{F})-\bm{G}(\bm{F})\rightarrow 0$。

因此,对于复合函数有:
$$
    \frac{d}{d\bm{T}}\bm{G}(\bm{F}(\bm{T}))*d\bm{T}
    =d\bm{G}(\bm{F}(\bm{T}))=\bm{G}'(\bm{F})*d\bm{F}(\bm{T})
    =\bm{G}'(\bm{F})*\bm{F}'(\bm{T})*d\bm{T}
$$
这对任何的$\bm{dT}=h\bm{Y},\  \forall\bm{Y},\  h\rightarrow 0$成立,因此可得
$$
    \frac{d}{d\bm{T}}\bm{G}(\bm{F}(\bm{T}))
    =\bm{G}'(\bm{F})*\bm{F}'(\bm{T})
$$

\subsection*{7.2}
$f'$是一个三阶张量,分量为$f'(\bm{P})_{ijk}=\pd{f(\bm{P})}{p_{ijk}}$,
则明显其非零的分量包括:
$$
    \begin{aligned}
        f'(\bm{P})_{111}&=2P_{111}                         \\
        f'(\bm{P})_{222}&=4P_{222}                         \\
        f'(\bm{P})_{123}&=f'(\bm{P})_{132}=P_{123}+P_{132} \\
        f'(\bm{P})_{313}&=f'(\bm{P})_{331}=P_{313}+P_{331} \\
    \end{aligned}
$$
这是对称的形式,由于$P$的对称性,还有其他等价的非对称表达式。


\subsection*{7.3} 

对于$\bm{B} \in \mathbb{R}^{n\times n}$,
不妨将某个基下面的一种混合分量记为$B^i_{\cdot j}$,其矩阵记为$[B]$,
明显,$\det(\bm{B})=\det([B])$。

考虑矩阵行列式按某一行或者某一列展开,则明显有:
$$
\pd{\det([B])}{B^i_{\cdot j}}=(-1)^{i+j}R_{ij}
$$
其中有$R_{ij}$是矩阵$[B]$的余子式,也就是去掉$i$行、$j$列后的行列式。
明显有[$\pd{\det([B])}{B^i_{\cdot j}}]=[B^*]\trans$,也就是这个偏导数矩阵就是
伴随矩阵的转置。

线性代数的结论给出:$[B^*][B]=[I]\det([B])$,因此有:
$$
\pd{\det(\bm{B})}{B^i_{\cdot k}}B^j_{\cdot k}=\delta_i^j\det(\bm{B})
$$
也就是
\begin{equation}
    \pd{\det(\bm{B})}{\bm{B}}\bm{B}\trans=\bm{I}\det(\bm{B})
    \label{eq:mid1}
\end{equation}

根据Hamilton-Caylay定理,有:

\begin{equation}
    \det(\bm{B})\bm{I}=I_n\bm{I}=\sum_{i=1}^n{(-1)^{n-1}I_{n-i}\bm{B}^n}
    \label{eq:HC}
\end{equation}

其中$I_0=1$,$I_1,I_2,\dots I_n$是特征多项式的系数,也就是主不变量。
将\eqref{eq:HC}转置后代入\eqref{eq:mid1},则有:

\begin{equation}
    \pd{\det(\bm{B})}{\bm{B}}\cdot\bm{B}\trans=\sum_{i=1}^n{(-1)^{n-1}I_{n-i}(\bm{B}\trans)^n}
\end{equation}

对于非奇异的$\bm{B}$,两端右点积$\bm{B}^{-\mathrm{T}}$,即有
\begin{equation}
    \pd{\det(\bm{B})}{\bm{B}}=\sum_{i=1}^n{(-1)^{n-1}I_{n-i}(\bm{B}\trans)^{n-1}}
    \label{eq:result}
\end{equation}

又由于,在$\mathbb{R}^{n\times n}$中,奇异二阶张量的邻域一定有非奇异二阶张量,而
根据行列式函数的无穷阶连续可导性质,取非奇异数列$\bm{B}_k$趋近于每个奇异的二阶张量都有\eqref{eq:result},
因此本式对所有$\mathbb{R}^{n\times n}$的二阶张量都成立。

特别地,对三维情况有:

$$
\pd{\det(\bm{B})}{\bm{B}}=I_2\bm{I}-I_1\bm{B}\trans+(\bm{B}\trans)^2
$$

\end{document}