%!TEX program = xelatex
\documentclass[UTF8,zihao=5]{ctexart}


\title{{\bfseries 第8次作业}}
\author{周涵宇 2022310984}
\date{}

\usepackage[a4paper]{geometry}
\geometry{left=0.75in,right=0.75in,top=1in,bottom=1in}

\usepackage[
UseMSWordMultipleLineSpacing,
MSWordLineSpacingMultiple=1.5
]{zhlineskip}

\usepackage{fontspec}
\setmainfont{Times New Roman}
% \setmonofont{JetBrains Mono}
\setCJKmainfont{仿宋}[AutoFakeBold=true]
\setCJKsansfont{黑体}[AutoFakeBold=true]

\usepackage{bm}

\usepackage{amsmath,amsfonts}
\usepackage{array}

% \newcommand{\bm}[1]{{\mathbf{#1}}}

\newcommand{\trans}[0]{^\mathrm{T}}
\newcommand{\tran}[1]{#1^\mathrm{T}}
\newcommand{\hermi}[0]{^\mathrm{H}}
\newcommand{\conj}[1]{\overline{#1}}

\newcommand*{\av}[1]{\left\langle{#1}\right\rangle}

\newcommand*{\avld}[1]{\frac{\overline{D}#1}{Dt}}

\newcommand*{\pd}[2]{\frac{\partial #1}{\partial #2}}

\newcommand*{\pdcd}[3]
{\frac{\partial^2 #1}{\partial #2 \partial #3}}


\begin{document}

\maketitle

\subsection*{8.1}

对系统采用虚位移法,则对位置进行变分。

本题中记号的$i,j,k$等指标都为$1,2$范围变化,而非三维,采用爱因斯坦求和。

记原曲面在三维欧氏空间的位置矢量$\bm{r}$,
变分的位移
$\delta\bm{r}=\zeta \bm{n} + \xi^i \bm{g}_i$,
其中$\xi^1,\xi^2,\zeta$是变分的小量,为曲面坐标的函数。
那么能量变分为:

$$
    \delta E
    =
    \gamma_{LV}\int_{\Sigma_{LV}}{\delta dA}
    +
    (\gamma_{SL}-\gamma_{SV})
    \int_{\Sigma_{SL}}{\delta dA}
    +
    \int_{\Omega_{L}}{P\delta dV}
    +
    \tau \int_{\partial\Sigma_{LV}}{\delta ds}
$$

其中,在两个曲面的交界处,坐标系和法向并不相同,但是变分的位置矢量
是相同的矢量。

考虑
$$
    \begin{aligned}
        \delta\bm{g}_i
         & =\delta\pd{\bm{r}}{x^i}
        =\pd{}{x^i}(\zeta \bm{n} + \xi^j \bm{g}_j) \\
         & =\zeta_{,i}\bm{n} + \xi^j_{,i}\bm{g}_j
        + \zeta \bm{n}_{,i} + \xi^j \bm{g}_{j,i}   \\
         & =\zeta_{,i}\bm{n} + \xi^j_{,i}\bm{g}_j
        -\zeta b_{ij}\bm{g}^j +
        \xi^j \Gamma_{ij,k}\bm{g}^k + \xi^j b_{ij}\bm{n}
    \end{aligned}
$$
因此度量系数变分为:
$$
    \begin{aligned}
        \delta g_{ij}
         & = \delta\bm{g}_i\cdot\bm{g}_j + \delta\bm{g}_j\cdot\bm{g}_i \\
         & = \xi^k_{,i}g_{kj}
        -\zeta b_{ij} +
        \xi^k \Gamma_{ik,j} +
        \xi^k_{,j}g_{ki}
        -\zeta b_{ji} +
        \xi^k \Gamma_{jk,i}                                            \\
         & = -2\zeta b_{ij}
        + (\xi^k_{,i}+ \xi^m \Gamma_{im}^k)g_{kj}
        + (\xi^k_{,j}+ \xi^m \Gamma_{jm}^k)g_{ki}                      \\
         & =
        -2\zeta b_{ij}
        + \xi^k_{;i}g_{kj}
        + \xi^k_{;j}g_{ki}                                             \\
    \end{aligned}
$$
考虑计算面元的变分,首先计算度量系数行列式的变分。
根据行列式的导数结论以及度量系数的对称性,
以及已知$g^{ij}g_{jk}=\delta^i_k$
$$
    \begin{aligned}
        \delta g & =
        g g^{ij}\delta g_{ij}                \\
                 & =
        -2g\zeta b_{ij}g^{ij} + 2g\xi^k_{;k} \\
                 & =
        -4gH\zeta + 2g\xi^k_{;k}
    \end{aligned}
$$
其中$2H=b^i_{\cdot i}$是平均曲率。则面元的变分是
$$
    \delta dA = (\delta\sqrt{g})dx^1dx^2
    =\frac{1}{2\sqrt{g}}\delta g dx^1dx^2=
    (-2H\zeta + \xi^k_{;k})dA
$$

接下来考虑体积变分,根据体积元,有
% $$
%     \begin{aligned}
%         \delta dV & =
%         \delta (\frac{1}{3}\bm{r}\cdot\bm{n}dA) \\
%                   & =
%         \frac{1}{3}dx^1dx^2\left(
%             \delta \bm{r}\cdot \bm{n}\sqrt{g}
%             +
%             \delta \bm{n}\cdot \bm{r}\sqrt{g}
%             +
%             \delta\sqrt{g}\bm{r}\cdot\bm{n}
%         \right)\\
%         &=
%         \frac{1}{3}\left(
%             \zeta dA
%             +
%             \delta \bm{n}\cdot \bm{r}dA
%             +
%             \bm{r}\cdot\bm{n} \delta dA
%         \right)
%     \end{aligned}
% $$
%! 锥形体积元有附加项,此处没有推导
$$
    \delta dV  = \bm{n}\cdot \delta \bm{r} dA =  \zeta dA
$$

考虑曲面上的曲线元的变分,对于欧氏空间中的一般曲线有:
$$
    \delta ds =
    (\frac{d}{ds}(\delta \bm{r} \cdot \bm{t})
    -
    \kappa\delta\bm{r} \cdot  \bm{n}_c)ds
$$

当变分只在曲面内时,则:
$$
    \delta ds =
    (\frac{d}{ds}(\delta \bm{r} \cdot \bm{t})
    -
    \kappa_g\delta\bm{r} \cdot  \bm{m})ds
$$
%! 此处应当有形式化推导
上式对于环线第一项积分为0。

其中$\bm{m}=\bm{t}\times\bm{n}$是切平面内的曲线法向,$\bm{t}$是切向。

综上,使变分在$\Sigma_{SL}$没有法向分量,则有

$$
    \delta E
    =
    \gamma_{LV}\int_{\Sigma_{LV}}{(-2H\zeta + \xi^k_{;k})dA}
    +
    (\gamma_{SL}-\gamma_{SV})
    \int_{\Sigma_{SL}}{(\xi^k_{;k})dA}
    +
    \int_{\Sigma_{LV}}{\Delta P \zeta dA}
    -
    \tau \int_{\partial\Sigma_{L}}{\kappa_g\delta\bm{r} \cdot \bm{m}ds}
$$

注意到曲面上(也就是黎曼空间内)的散度积分定理(对于仅有面内分量的张量):
%! 推导
$$
    \int_\Sigma{\xi^k_{;k}dA}=\int_{\partial\Sigma}
    {\hat{\delta\bm{r}}\cdot \bm{l} ds}
    =\int_{\partial\Sigma}
    {\delta\bm{r}\cdot \bm{l} ds}
$$
其中$l$是单位向量,其位于切平面内且与边界曲线的切向量垂直,对于
单连通曲面定义为向外。
其中$\hat{\delta\bm{r}}=\xi^i\bm{g}_i$是变分向量
$\delta\bm{r}$在切平面内的投影。由于法向分量和$\bm{l}$垂直,
即不影响环上的积分结果。

考虑到变分中边界线的面内法向$\bm{m}=-\bm{l}_{SL}$
(因为定义$\bm{m}$朝内,也就是边界线的曲率为正时液滴外凸)。


则上式可化简为:

$$
    \delta E
    =
    \int_{\Sigma_{LV}}{\zeta(-2\gamma_{LV}H + \Delta P) dA}
    +
    \int_{\partial\Sigma_{SL}}{\left[
            (\gamma_{SL}-\gamma_{SV})\bm{l}_{SL}
            +
            \gamma_{LV}\bm{l}_{LV}
            +
            \tau\kappa_g \bm{l}_{SL}
            \right]\cdot \delta\bm{r}
        ds}
$$

考虑到由于变分不离开固体面(也就是在边界切向量和$\bm{l}_{SL}$的平面内。
定义两个面边缘的法向向量$\bm{l}_{SL},\bm{l}_{LV}$夹角
为$\theta$,也就是液滴内界面切平面的夹角,根据射影几何有关系:
$$
    \delta \bm{r}\cdot \bm{l}_{LV} = \cos\theta \delta\bm{r}\cdot\bm{l}_{SL}
$$
也就是说$\delta\bm{r}$在固体面切空间内,上式变成:

$$
    \delta E
    =
    \int_{\Sigma_{LV}}{\zeta(-2\gamma_{LV}H + \Delta P) dA}
    +
    \int_{\partial\Sigma_{SL}}{\left[
            (\gamma_{SL}-\gamma_{SV})
            +
            \gamma_{LV}\cos \theta
            +
            \tau\kappa_g
            \right]\bm{l}_{SL}\cdot \delta\bm{r}
        ds}
$$
根据能量变分是0,以及变分量$\delta \bm{r}$的任意性,则两部分积分分别有:

$$
    \Delta P = 2H\gamma_{LV}
$$
$$
    \cos \theta = \frac{\gamma_{SV}-\gamma_{SL}}{\gamma_{LV}} - \frac{\tau\kappa_g}{\gamma_{LV}}
$$

\subsection*{8.2}

根据Young-Laplace方程可知平衡为:

$$
    2H=\frac{\rho gz}{\gamma}
$$

记$l_c=\sqrt{\frac{\gamma}{\rho g}}$,且根据拉伸直曲面的曲率可知:

$$
    \frac{d^2x}{dz^2}\left[1+\left(\frac{dx}{dz}\right)^2\right]^{-3/2} = \frac{z}{l_c^2}
$$

归一化为$\xi = x/l_c, w = z/l_c$,记$\frac{d\xi}{dw}=\xi', \frac{d^2\xi}{dw^2}=\xi''$

则解

$$
    \xi'' \left[1+\left(\xi'\right)^2\right]^{-3/2} = w
$$

根据$\xi(0)=+\infty$积分可得

$$
    \xi = c + \ln\frac{2 + \sqrt{4-w^2}}{w} - \sqrt{4-w^2}
$$

以及

$$
    \xi' = \frac{w^2-2}{w\sqrt{4-w^2}}
$$

可见$w=2$的深度时液面切平面是水平的,再深就没有解了;
同时$w=\sqrt{2}$时是竖直的,为最左端,因此记最左端为$\xi_0$,
最下端极限位置为$\xi_0+\lambda_{max}$,有$\lambda_{max}=\sqrt{2}-\ln(\sqrt{2}+1)$。

则
$$
    \xi = \xi_0 + \lambda_{max} + \ln\frac{2 + \sqrt{4-w^2}}{w} - \sqrt{4-w^2}
$$

在考虑圆柱外形的联系,如图,在归一化的坐标系应用几何关系有:
$$
    \begin{aligned}
        \xi' & =-\cot(\phi-\alpha)   \\
        h    & = w + r + r\cos\alpha \\
        \xi  & =  r\sin\alpha
    \end{aligned}
$$
给定圆柱尺寸和物性,上面的方程中$r,\cos\phi$是给定的,
$h,\alpha$都未知,同时接触点的$w$是未知的,常数$\xi_0$是未知的。

因此共三个方程,四个未知数,
可任选一个为自变量,如圆柱深度$h$,在每个位置求解非线性方程组,得到其他三个值,即可知道
页面和圆柱的具体关系。如果给定圆柱受到液体的合力大小,可解出一个确定的位置。

同时可知,不管圆柱有多深,没有与圆柱接触的部分的液面,形状与圆柱位置无关,只与无量纲长度$l_c$有关。

\end{document}