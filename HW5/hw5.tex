%!TEX program = xelatex
\documentclass[UTF8,zihao=5]{ctexart}


\title{{\bfseries 第五次作业}}
\author{周涵宇 2018011600}
\date{}

\usepackage[a4paper]{geometry}
\geometry{left=0.75in,right=0.75in,top=1in,bottom=1in}

\usepackage[
UseMSWordMultipleLineSpacing,
MSWordLineSpacingMultiple=1.5
]{zhlineskip}

\usepackage{fontspec}
\setmainfont{Cambria Math}
% \setmonofont{JetBrains Mono}
\setCJKmainfont{仿宋}[AutoFakeBold=true]
\setCJKsansfont{黑体}[AutoFakeBold=true]

\usepackage{bm}
\usepackage{amsmath,amsfonts}
\usepackage{array}

\newcommand{\trans}[0]{^\mathrm{T}}
\newcommand{\tran}[1]{#1^\mathrm{T}}
\newcommand{\hermi}[0]{^\mathrm{H}}

\newcommand*{\av}[1]{\left\langle{#1}\right\rangle}

\newcommand*{\avld}[1]{\frac{\overline{D}#1}{Dt}}

\newcommand*{\pd}[2]{\frac{\partial #1}{\partial #2}}

\newcommand*{\pdcd}[3]
{\frac{\partial^2 #1}{\partial #2 \partial #3}}


\begin{document}

\maketitle

\subsection*{1}

\begin{equation*}
    \begin{bmatrix}
        \bm{g_1}&\bm{g_2}&\bm{g_3}&\bm{g_4}
    \end{bmatrix}
    =
    \begin{bmatrix}
        \bm{e_1}&\bm{e_2}&\bm{e_3}&\bm{e_4}
    \end{bmatrix}
    \begin{bmatrix}
        5&1&-1&-2\\
        3&2& 3& 3\\
        1&0& 4& 0\\
        0&1& 0& 5\\
    \end{bmatrix}   
\end{equation*}

由于$\bm{g_i\cdot g^j}=\delta_{ij}$

\begin{equation*}
    \begin{bmatrix}
        \bm{g^1}\\\bm{g^2}\\\bm{g^3}\\\bm{g^4}
    \end{bmatrix}
    =
    \left[\begin{array}{cccc} \frac{1}{3} & -\frac{1}{3} & \frac{1}{3} & \frac{1}{3}\\ -\frac{15}{28} & \frac{5}{4} & -\frac{15}{14} & -\frac{27}{28}\\ -\frac{1}{12} & \frac{1}{12} & \frac{1}{6} & -\frac{1}{12}\\ \frac{3}{28} & -\frac{1}{4} & \frac{3}{14} & \frac{11}{28} \end{array}\right]
    \begin{bmatrix}
        \bm{e_1}\\\bm{e_2}\\\bm{e_3}\\\bm{e_4}
    \end{bmatrix}
\end{equation*}

即
$$
\bm{g^1}=
 \frac{1}{3}\bm{e_1}
-\frac{1}{3}\bm{e_2}
+\frac{1}{3}\bm{e_3}
+\frac{3}{3}\bm{e_4}
$$
$$
\bm{g^2}=
-\frac{15}{28}\bm{e_1}
+\frac{5}{4}\bm{e_2}
-\frac{15}{14}\bm{e_3}
-\frac{27}{28}\bm{e_4}
$$
$$
\bm{g^3}=
-\frac{1}{12}\bm{e_1}
+\frac{1}{12}\bm{e_2}
+\frac{1}{6}\bm{e_3}
-\frac{1}{12}\bm{e_4}
$$
$$
\bm{g^4}=
 \frac{3}{28}\bm{e_1}
-\frac{1}{4}\bm{e_2}
+\frac{3}{14}\bm{e_3}
+\frac{11}{28}\bm{e_4}
$$

\subsection*{2}

已知,在正交坐标系$e_{i'}$

$$
I_1=trB=B^{i'}_{\cdot i'}=B^{\cdot i'}_{i'}
$$

坐标转换即为
$$
I_1=B^{j}_{\cdot k}\beta_{j}^{i'}\beta_{i'}^{k}
=B^{j}_{\cdot k}\delta_{j}^{k}=B^{k}_{\cdot k}
$$

同理也有

$$
I_1=B^{\cdot i'}_{i'}=B^{\cdot i}_{i}
$$

根据指标升降关系和度量张量的对称性同时有
$$
I_1=B^{ij}g_{ij}==B_{ij}g^{ij}
$$



$$
I_2=\frac{1}{2}(tr^2B-trB^2)=
\frac{1}{2}\left(
    B^{i'}_{\cdot i'}B^{j'}_{\cdot j'}
    -B^{i'}_{\cdot j'}B^{j'}_{\cdot i'}
\right)
$$

由第一不变量的结论并坐标变换

$$
I_2=\frac{1}{2}(tr^2B-trB^2)=
\frac{1}{2}\left(
    B^{i}_{\cdot i}B^{j}_{\cdot j}
    -B^{k}_{\cdot l}B^{m}_{\cdot n}
    \beta_{k}^{i'}\beta_{j'}^{l}
    \beta_{m}^{j'}\beta^{n}_{i'}
\right)
=
\frac{1}{2}\left(
    B^{i}_{\cdot i}B^{j}_{\cdot j}
    -B^{k}_{\cdot l}B^{m}_{\cdot n}
    \delta_{k}^{n}
    \delta_{m}^{l}
\right)
=
\frac{1}{2}\left(
    B^{i}_{\cdot i}B^{j}_{\cdot j}
    -B^{k}_{\cdot k}B^{m}_{\cdot m}
\right)
$$

由正交坐标

$$
I_3=e^{i'j'k'}e_{r's't'}B^{r'}_{\cdot i'}B^{s'}_{\cdot j'}B^{t'}_{\cdot k'}
=\epsilon^{i'j'k'}\epsilon_{r's't'}B^{r'}_{\cdot i'}B^{s'}_{\cdot j'}B^{t'}_{\cdot k'}
$$

以上$\epsilon$是Eddington张量的分量,其满足坐标变换规则,简单坐标变换即有

$$
I_3
=\epsilon^{ijk}\epsilon_{rst}B^{r}_{\cdot i}B^{s}_{\cdot j}B^{t}_{\cdot k}
$$

同时代入指标升降发现$\epsilon$和$B$的上下标缩并可任意互换上下位置,则

$$
I_3
=\epsilon_{ijk}\epsilon^{rst}B_{r}^{\cdot i}B_{s}^{\cdot j}B_{t}^{\cdot k}
=\epsilon_{ijk}\epsilon_{rst}B^{ri}B^{sj}B^{tk}
=\epsilon^{ijk}\epsilon^{rst}B_{ri}B_{sj}B_{tk}
$$

由于$\epsilon_{ijk}=\sqrt{g}e_{ijk},\epsilon^{ijk}=\frac{1}{\sqrt{g}}e^{ijk}$

$$
I_3
=e^{ijk}e_{rst}B^{r}_{\cdot i}B^{s}_{\cdot j}B^{t}_{\cdot k}=\det{B^i_{\cdot j}}
$$

$$
I_3
=\epsilon_{ijk}\epsilon^{rst}B_{r}^{\cdot i}B_{s}^{\cdot j}B_{t}^{\cdot k}=\det{B_i^{\cdot j}}
$$

$$
I_3
=g\epsilon_{ijk}\epsilon_{rst}B^{ri}B^{sj}B^{tk}=g\det{B^{ij}}
$$

$$
I_3
=\frac{1}{g}\epsilon^{ijk}\epsilon^{rst}B_{ri}B_{sj}B_{tk}=\frac{1}{g}\det{B_{ij}}
$$

\subsection*{3}

由于
$x^1=x^{1'}\sin{x^{3'}}$,
$x^2=x^{2'}$,
$x^3=x^{1'}\cos{x^{3'}}$
则

$$
\beta^{i}_{j'}
=
\frac{\partial x^i}{\partial x^{j'}}
=
\begin{bmatrix}
    \sin{x^{3'}}&0&x^{1'}\cos{x^{3'}}\\
    0&1&0\\
    \cos{x^{3'}}&0&-x^{1'}\sin{x^{3'}}\\
\end{bmatrix}_{ij}
$$

$$
\beta^{i'}_{j}
=
\begin{bmatrix}
    \sin{x^{3'}}&0&\cos{x^{3'}}\\
    0&1&0\\
    \cos{x^{3'}}/x^{1'}&0&-\sin{x^{3'}}/x^{1'}\\
\end{bmatrix}_{ij}
$$

\subsection*{4}

已知第二类Christoffel符号和度量张量分量:

$$
\Gamma^p_{ij}=\frac{1}{2}g^{kp}\left(
\pd{g_{ik}}{x^j}+\pd{g_{jk}}{x^i}
-\pd{g_{ij}}{x^k}
\right)
$$

则已知Christoffel符号在斜角直线坐标下是0数组,但其在其他坐标下不是0,因此不是满足张量
与坐标不变的关系(0张量和非零一定不等),因此Christoffel符号不是张量分量。

现假设分量对坐标的偏导$\pd{u^i}{x^j}$是张量分量,则由协变导数的计算公式:

$$
u^i_{;j}=\pd{u^i}{x^j}+u^{m}\Gamma^i_{jm}
$$

由于协变导数是梯度的分量,其是张量分量,由于加和关系,假设坐标偏导是张量分量则有
$u^{m}\Gamma^i_{jm}$是张量分量,由于$u^m$是张量分量,根据商法则即$\Gamma^i_{jm}$是
张量分量。这与上面的结论矛盾,因此分量对坐标的偏导不是张量分量。

同样,按照相同思路讨论公式

$$
u_{i;j}=\pd{u_i}{x^j}-u_{m}\Gamma^m_{ji}
$$

则同样得到$\pd{u_i}{x^j}$也不是张量分量。

\subsection*{5}

可知在柱坐标中:

$$
g_{11}=1, g_{22}=x^{1}x^{1}, g_{33}=1
$$

其余度量分量都是0

则可知
$$
\Gamma_{ij,k}=\frac{1}{2}\left(
\pd{g_{ik}}{x^j}+\pd{g_{jk}}{x^i}
-\pd{g_{ij}}{x^k}
\right)
$$


$$
\Gamma_{22,1}=\frac{1}{2}\left(
    -\pd{g_{22}}{x^1}
    \right)=-x^{1}
$$

$$
\Gamma_{21,2}=\Gamma_{12,2}=\frac{1}{2}\left(
\pd{g_{22}}{x^1}
\right)=x^{1}
$$

同理,在球坐标中已知:
$$
g_{11}=1, 
g_{22}=x^{1}x^{1}\sin^2{x^{3}}, 
g_{33}=x^{1}x^{1}
$$

则可知
$$
\Gamma_{ij,k}=\frac{1}{2}\left(
\pd{g_{ik}}{x^j}+\pd{g_{jk}}{x^i}
-\pd{g_{ij}}{x^k}
\right)
$$

其中

则可知
$$
\Gamma_{22,1}=\frac{1}{2}\left(
-\pd{g_{22}}{x^1}
\right)=-x^{1}\sin^2{x^{3}}
$$

$$
\Gamma_{12,2}=\Gamma_{21,2}=\frac{1}{2}\left(
    \pd{g_{22}}{x^1}
    \right)=x^{1}\sin^2{x^{3}}
$$

$$
\Gamma_{22,3}=\frac{1}{2}\left(
-\pd{g_{22}}{x^3}
\right)=-x^{1}x^{1}\sin{x^{3}}\cos{x^{3}}
$$

$$
\Gamma_{23,2}=\Gamma_{32,2}=\frac{1}{2}\left(
    \pd{g_{22}}{x^1}
    \right)=x^{1}x^{1}\sin{x^{3}}\cos{x^{3}}
$$

$$
\Gamma_{33,1}=\frac{1}{2}\left(
-\pd{g_{33}}{x^1}
\right)=-x^{1}
$$

$$
\Gamma_{13,3}=\Gamma_{31,3}=\frac{1}{2}\left(
    \pd{g_{33}}{x^1}
    \right)=x^{1}
$$

\subsection*{6.}

\begin{equation*}
    \begin{split}
        \sqrt{g}_{,i}
        =&\bm{[g_1,g_2,g_3]}_{,i}
        \\=&
        \bm{(\pd{g_{1}}{x_i}\times g_2)\cdot g_3}
        +\bm{(g_{1}\times \pd{g_{2}}{x_i})\cdot g_3}
        +\bm{(g_{1}\times g_2)\cdot \pd{g_{3}}{x_i}}
        \\=&
        \bm{(\Gamma^{k}_{1i}g_k\times g_2)\cdot g_3}
        +\bm{(g_{1}\times \Gamma^{k}_{2i}g_k)\cdot g_3}
        +\bm{(g_{1}\times g_2)\cdot \Gamma^{k}_{3i}g_k}
        \\=&
        \bm{(\Gamma^{1}_{1i}g_1\times g_2)\cdot g_3}
        +\bm{(g_{1}\times \Gamma^{2}_{2i}g_2)\cdot g_3}
        +\bm{(g_{1}\times g_2)\cdot \Gamma^{3}_{3i}g_3}
        \\=&
        \Gamma^{j}_{ji}\sqrt{g}
    \end{split}
\end{equation*}

则

$$
\Gamma^{j}_{ji}=\frac{1}{\sqrt{g}}\pd{\sqrt{g}}{x_i}
=\pd{\ln{\sqrt{g}}}{x_i}
$$









\end{document}